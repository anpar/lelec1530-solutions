\documentclass[frenchb,DIV=14]{scrartcl}

\usepackage[utf8x]{inputenc}
\usepackage[T1]{fontenc}
\usepackage{lmodern}
\usepackage{microtype}
\usepackage{xspace}
\usepackage[binary-units=true]{siunitx}
%\usepackage{color}
%\usepackage[usenames,dvipsnames,svgnames,table]{xcolor}
\usepackage{graphicx}
\usepackage{hyperref}
\usepackage{todonotes}
\usepackage{epstopdf}
\usepackage{array}
\usepackage{multicol}
\usepackage{multirow}
\usepackage{tabularx} % tabular with automatic line-break
\newcolumntype{Y}{>{\centering\arraybackslash}X} % centered column
\usepackage{ifthen}
\usepackage{amsmath}
\usepackage{grffile} % better name handling with graphicx
\usepackage{currfile} % provides relative file inclusion for tikzscale

\usepackage{tikz}
\usepackage{pgfplots}
\usepackage{tikz-3dplot}
\usepackage{pgfplots}
\usepackage{tikzscale}
\pgfplotsset{compat=newest}
\usetikzlibrary{plotmarks}
\usepackage{rotating}

\usepackage[french]{babel}
% Listing
% always put it after babel
% http://tex.stackexchange.com/questions/100717/code-in-lstlisting-breaks-document-compile-error
\usepackage{listings}

\definecolor{mygreen}{rgb}{0,0.6,0}
\definecolor{mygray}{rgb}{0.5,0.5,0.5}
\definecolor{mymauve}{rgb}{0.58,0,0.82}
\lstset{ %
  language=Matlab,
  backgroundcolor=\color{white},   % choose the background color; you must add \usepackage{color} or \usepackage{xcolor}
  basicstyle=\footnotesize,        % the size of the fonts that are used for the code
  breakatwhitespace=false,         % sets if automatic breaks should only happen at whitespace
  breaklines=true,                 % sets automatic line breaking
  captionpos=b,                    % sets the caption-position to bottom
  commentstyle=\color{mygreen},    % comment style
  deletekeywords={...},            % if you want to delete keywords from the given language
  escapeinside={\%*}{*)},          % if you want to add LaTeX within your code
  extendedchars=true,              % lets you use non-ASCII characters; for 8-bits encodings only, does not work with UTF-8
  frame=single,	                   % adds a frame around the code
  keepspaces=true,                 % keeps spaces in text, useful for keeping indentation of code (possibly needs columns=flexible)
  keywordstyle=\color{blue},       % keyword style
  otherkeywords={*,...},           % if you want to add more keywords to the set
  numbers=none,                    % where to put the line-numbers; possible values are (none, left, right)
  numbersep=5pt,                   % how far the line-numbers are from the code
  numberstyle=\tiny\color{mygray}, % the style that is used for the line-numbers
  rulecolor=\color{black},         % if not set, the frame-color may be changed on line-breaks within not-black text (e.g. comments (green here))
  showspaces=false,                % show spaces everywhere adding particular underscores; it overrides 'showstringspaces'
  showstringspaces=false,          % underline spaces within strings only
  showtabs=false,                  % show tabs within strings adding particular underscores
  stepnumber=2,                    % the step between two line-numbers. If it's 1, each line will be numbered
  stringstyle=\color{mymauve},     % string literal style
  tabsize=2,	                   % sets default tabsize to 2 spaces
  title=\lstname                   % show the filename of files included with \lstinputlisting; also try caption instead of title
}

\newcommand{\matlab}{\textsc{Matlab}}

% Math symbols
\usepackage{amsmath}
\usepackage{amssymb}
\usepackage{amsthm}
\DeclareMathOperator*{\argmin}{arg\,min}
\DeclareMathOperator*{\argmax}{arg\,max}


% Sets
\newcommand{\Z}{\mathbb{Z}}
\newcommand{\R}{\mathbb{R}}
\newcommand{\Rn}{\R^n}
\newcommand{\Rnn}{\R^{n \times n}}
\newcommand{\C}{\mathbb{C}}
\newcommand{\K}{\mathbb{K}}
\newcommand{\Kn}{\K^n}
\newcommand{\Knn}{\K^{n \times n}}

% Unit vectors
\usepackage{esint}
\usepackage{esvect}
\newcommand{\kmath}{k}
\newcommand{\xunit}{\hat{\imath}}
\newcommand{\yunit}{\hat{\jmath}}
\newcommand{\zunit}{\hat{\kmath}}
\newcommand{\uunit}{\hat{\umath}}

% rot & div & grad & lap
\DeclareMathOperator{\newdiv}{div}
\newcommand{\divn}[1]{\nabla \cdot #1}
\newcommand{\rotn}[1]{\nabla \times #1}
\newcommand{\grad}[1]{\nabla #1}
\newcommand{\gradn}[1]{\nabla #1}
\newcommand{\lap}[1]{\nabla^2 #1}


% Elec
\newcommand{\B}{\vec B}
\newcommand{\E}{\vec E}
\newcommand{\EMF}{\mathcal{E}}
\newcommand{\perm}{\varepsilon} % permittivity

\newcommand{\bigoh}{\mathcal{O}}
\newcommand\eqdef{\triangleq}

\DeclareMathOperator{\newdiff}{d} % use \dif instead
\newcommand{\dif}{\newdiff\!}
\newcommand{\fpart}[2]{\frac{\partial #1}{\partial #2}}
\newcommand{\ffpart}[2]{\frac{\partial^2 #1}{\partial #2^2}}
\newcommand{\fdpart}[3]{\frac{\partial^2 #1}{\partial #2\partial #3}}
\newcommand{\fdif}[2]{\frac{\dif #1}{\dif #2}}
\newcommand{\ffdif}[2]{\frac{\dif^2 #1}{\dif #2^2}}
\newcommand{\constant}{\ensuremath{\mathrm{cst}}}

%\KOMAoptions{DIV=last}

\usepackage{circuitikz}
\usetikzlibrary{babel}
\ctikzset{tripoles/mos style/arrows}

\makeatletter
\pgfcircdeclaremos{pmos}{
          \anchor{S}{
            \northeast
          }
          \anchor{source}{
            \northeast
          }
          \anchor{D}{
            \northeast
            \pgf@y=-\pgf@y
          }
          \anchor{drain}{
            \northeast
            \pgf@y=-\pgf@y
          }
}{%
            \pgfpathmoveto{\pgfpoint{\pgf@circ@res@right}{\pgf@circ@res@up}}
            \pgfpathlineto{\pgfpoint{\pgf@circ@res@right}
                {\pgfkeysvalueof{/tikz/circuitikz/tripoles/pmos/gate height}\pgf@circ@res@up}}
            \pgfpathlineto{\pgfpoint
                {\pgfkeysvalueof{/tikz/circuitikz/tripoles/pmos/base width}\pgf@circ@res@left}
                {\pgfkeysvalueof{/tikz/circuitikz/tripoles/pmos/gate height}\pgf@circ@res@up}}
            \pgfusepath{draw}

        \ifpgf@circuit@mos@arrows
            \pgfscope             
            \pgfslopedattimetrue 
            \pgfallowupsidedownattimetrue
            \pgfresetnontranslationattimefalse
            \pgftransformlineattime{.4}{%
                \pgfpoint%
                    {\pgf@circ@res@right}%
                    {\pgfkeysvalueof{/tikz/circuitikz/tripoles/pmos/gate height}\pgf@circ@res@up}%
            }{%
                \pgfpoint
                    {\pgfkeysvalueof{/tikz/circuitikz/tripoles/pmos/gate width}\pgf@circ@res@left}%
                    {\pgfkeysvalueof{/tikz/circuitikz/tripoles/pmos/gate height}\pgf@circ@res@up}%
            }
            \pgfnode{currarrow}{center}{}{}{\pgfusepath{stroke}}
            \endpgfscope
        \fi

            \pgfscope
            \pgfpathmoveto{\pgfpoint
                {\pgfkeysvalueof{/tikz/circuitikz/tripoles/pmos/base width}\pgf@circ@res@left}
                {\pgfkeysvalueof{/tikz/circuitikz/tripoles/pmos/base height}\pgf@circ@res@up}}
            \pgfpathlineto{\pgfpoint
                {\pgfkeysvalueof{/tikz/circuitikz/tripoles/pmos/base width}\pgf@circ@res@left}
                {\pgfkeysvalueof{/tikz/circuitikz/tripoles/pmos/base height}\pgf@circ@res@down}}
            \pgfsetlinewidth{2\pgflinewidth}
            \pgfusepath{draw}
            \endpgfscope

            \pgfpathmoveto{\pgfpoint
                {\pgfkeysvalueof{/tikz/circuitikz/tripoles/pmos/base width}\pgf@circ@res@left}
                {\pgfkeysvalueof{/tikz/circuitikz/tripoles/pmos/gate height}\pgf@circ@res@down}}
            \pgfpathlineto{\pgfpoint{\pgf@circ@res@right}
                {\pgfkeysvalueof{/tikz/circuitikz/tripoles/pmos/gate height}\pgf@circ@res@down}}
            \pgfpathlineto{\pgfpoint{\pgf@circ@res@right}{\pgf@circ@res@down}}      

            \pgfpathmoveto{\pgfpoint
                {\pgfkeysvalueof{/tikz/circuitikz/tripoles/pmos/gate width}\pgf@circ@res@left}
                {\pgfkeysvalueof{/tikz/circuitikz/tripoles/pmos/gate height}\pgf@circ@res@up}}
            \pgfpathlineto{\pgfpoint
                {\pgfkeysvalueof{/tikz/circuitikz/tripoles/pmos/gate width}\pgf@circ@res@left}
                {\pgfkeysvalueof{/tikz/circuitikz/tripoles/pmos/gate height}\pgf@circ@res@down}}


            \pgfpathmoveto{\pgfpoint
                {\pgfkeysvalueof{/tikz/circuitikz/tripoles/pmos/gate width}\pgf@circ@res@left}
                {\pgf@circ@res@up+\pgf@circ@res@down}}
            \pgfpathlineto{\pgfpoint{\pgf@circ@res@left}{\pgf@circ@res@up+\pgf@circ@res@down}}
            \pgfusepath{draw}

%           \pgfpathcircle{\pgfpoint
%               {\pgfkeysvalueof{/tikz/circuitikz/tripoles/pmos/gate width}\pgf@circ@res@left - \pgfkeysvalueof{/tikz/circuitikz/nodes width}*\pgfkeysvalueof{/tikz/circuitikz/bipoles/length}}
%               {\pgf@circ@res@up+\pgf@circ@res@down}}{\pgfkeysvalueof{/tikz/circuitikz/nodes width}*\pgfkeysvalueof{/tikz/circuitikz/bipoles/length}}
%           \pgfusepath{draw,fill}      

}
\usetikzlibrary{intersections}
\usepgfplotslibrary{fillbetween}
\pgfdeclarelayer{bg}
\pgfsetlayers{bg,main}

\titlehead{}
\subject{LELEC1530}
\title{Séance 10 - Paires différentielles}
\subtitle{Solutions}
\author{\small Gaëtan \textsc{Cassiers} \and\small Antoine \textsc{Paris}}
\date{}

\begin{document}
\maketitle


\section*{Exercice 1: Paire différentielle simple}
On considère la paire différentielle de la Figure \ref{fig11-1} avec :

\begin{align*}
    k_n = k_p &= \SI{200}{\micro A/V^2} &
    V_A &= \SI{50}{V} &
    V_T &= \SI{1}{V} &
    V_{DD} &= \SI{5}{V} &
    V_1 = V_2 &= \SI{0}{V}
\end{align*}

\begin{figure}
    \centering
    \includegraphics[width=5cm]{figures/fig11-1.png}
    \caption{Exercice 1}
    \label{fig11-1}
\end{figure}

\begin{enumerate}
    \item  Pour $I_{ref}= \SI{10}{\micro A}$, calculez la dynamique de sortie de la paire, la transconductance des transistors d'entrée et la conductance d'Early des 4 transistors.
    \item  Calculez la résistance de sortie de la paire différentielle
    \item  Obtenez l'expression et la valeur du gain en tension de la paire par le principe de superposions.
    \item  Reproduisez les résultats de a, b et c pour $I_{ref}= \SI{100}{\micro A}$.
\end{enumerate}

\hspace{1cm}\hrule\hspace{1cm}

On note M1 et M2 les NMOS de gauche et droite respectivement, et M3 et M4 les PMOS
de gauche et droite respectivement.

\subsection*{Analyse DC}

Dans cette partie, on néglige l'effet Early. Le circuit est donc symétrique (les PMOS
sont identiques et en miroir de courant, les NMOS sont identiques et leur polarisation
est identique). On a donc $I_1 = I_2 = I_{ref}/2$ ($I_1$ et $I_2$ sont les courants dans
les branches de gauche et de droite, respectivement).

Soit $V_0$ la tension aux sources des NMOS, $V_{GS1}$ est donc donné
par $V_1-V_0$ et on a
\[I_1 = \frac{k_n}{2}\left(V_1-V_0-V_T\right)^2,\]
donc
\[V_{OV1} = V_1-V_0-V_T = \sqrt{\frac{2I_1}{k_n}} = \SI{0.22}{V}.\]

\paragraph{Dynamique de sortie}
$V_{OUT}$ doit être telle que les transistors restent
en régime saturé. Pour le NM0S: $V_{OUT} = V_{DS2} + V_0$.
Or, pour être en saturation, il faut $V_{DS2} \ge V_2 - V_0 - V_T$.
On trouve donc
\[V_{OUT} \ge V_2 - V_T = \SI{-1}{V}\].

Pour le PMOS: on calcule la tension de grille des PMOS:
\[I_1 = \frac{k_p}{2}\left(V_{DD}-V_{G4}-V_T\right)^2,\]
donc $V_{G4} = \SI{3.78}{V}$. Pour que le PMOS M4 soit en
saturation, il faut $V_{DD} - V_{OUT} \ge V_{SG} - V_T$.
On a donc
\[V_{OUT} \le V_{G,P}+V_T = \SI{4.78}{V}\].

\paragraph{Transconductance des transistors d'entrée}
\[g_m = g_{m,1} = g_{m,2} = \frac{2I_1}{V_{OV}} = \SI{44.7}{\micro S}\]

\paragraph{Conductance d'Early}
\[g_{d,1} = g_{d,2} = g_{d,3} = g_{d,4} = \frac{I_1}{V_A} = \SI{100}{nS}\]

\subsection*{Résistance de sortie}
En se rappellant qu'un transistor de transconductance $g_m$ monté en diode
(i.e. dont la grille est connectée au drain) est équivalent à une résistance
de valeur $1/g_m$, on obtient le schéma petit-signal suivant. Notons également
que la source de courant DC introduit une \emph{masse virtuelle} AC aux sources
de M1 et M2.
\begin{center}
\begin{circuitikz}
    \draw
    (1, 4) to[short] (-2, 4) to[R=$1/g_{m,3}$, i=$i_3$] (-2, 2) to[cI=$g_{m}v_1$] (-2, 0)
    to[short] (4, 0)
    to[R=$r_0$] (4, 2) to[R=$r_0$] (4, 4) to[short] (1, 4)
    to[cI=$i_3$] (1, 2) to[cI=$g_{m}v_2$] (1, 0)
    (1, 2) to[short, -*] (4, 2) to[short, -o] (5, 2) node[anchor=west] {$v_{out}$}
    (1, 0) node[ground] {}
    (1, 4) node[ground, yscale=-1] {};
\end{circuitikz}
\end{center}
Dans la branche de gauche du circuit, il faudrait en théorie ajouter les résistances
d'Early des transistors M1 et M3. Néanmoins, comme $\frac{1}{g_{m,3}} \ll (r_{0,1} || r_{0,2})$,
celles-ci peuvent-être négligées.

On calcule ensuite la résistance de sortie comme
\[R_{out} = \frac{v_t}{i_t}\bigg\rvert_{v_1=v_2=0}\]
avec $v_t$ une source de tension de test à la sortie et $i_t$
le courant courant entrant. On obtient finalement
\[R_{out} = \frac{r_0}{2}. \]

\subsection*{Gain différentiel}
Le gain différentiel est donné par $A = \frac{v_{out}}{v_{di}}$
avec $v_{di} = v_1 - v_2$. Par le principe de superposition,
on applique d'abord $v_1$ et on met $v_2$ à zéro:
\begin{center}
\begin{circuitikz}
    \draw
    (1, 4) to[short] (-2, 4) to[R=$1/g_{m,3}$, i=$i_3$] (-2, 2) to[cI=$g_{m}v_1$] (-2, 0)
    to[short] (4, 0)
    to[R=$r_0$] (4, 2) to[R=$r_0$] (4, 4) to[short] (1, 4)
    to[cI=$i_3$] (1, 2)
    (1, 2) to[short, -*] (4, 2) to[short, -o] (5, 2) node[anchor=west] {$v_{out}$}
    (1, 0) node[ground] {}
    (1, 4) node[ground, yscale=-1] {};
\end{circuitikz}
\end{center}
On a alors un courant $i_3 = g_m v_1$ qui passe dans la
résistance de sortie pour donner une tension $\frac{r_0}{2} g_m v_1$.

On applique ensuite $v_2$ et on met $v_1$ à zéro:
\begin{center}
\begin{circuitikz}
    \draw
    (1, 4) to[short] (-2, 4) to[R=$1/g_{m,3}$, i=$i_3$] (-2, 2) to[open] (-2, 0)
    to[short] (4, 0)
    to[R=$r_0$] (4, 2) to[R=$r_0$] (4, 4) to[short] (1, 4)
    to[cI=$i_3$] (1, 2) to[cI=$g_{m}v_2$] (1, 0)
    (1, 2) to[short, -*] (4, 2) to[short, -o] (5, 2) node[anchor=west] {$v_{out}$}
    (1, 0) node[ground] {}
    (1, 4) node[ground, yscale=-1] {};
\end{circuitikz}
\end{center}
On a alors un courant $-g_m v_2$ qui passe dans la résistance
de sortie pour donner une tension $-\frac{r_0}{2} g_m v_2$.

On a donc au total: $v_{out} = \frac{r_0}{2} g_m (v_1 - v_2) = \frac{r_0}{2} g_m v_{di}$.
Le gain est alors $A = \frac{r_0}{2}g_m = 224$.

\section*{Exercice 2 : Paire différentielle complexe}

On considère la paire différentielle complexe de la Figure~\ref{fig11-2} dans une technologie
canal long 1$\mu$m. Les paramètres technologiques sont dans la table 9.1 du cours. NMOS : 10/2,
PMOS : 20/2. 
Obtenez l'expression analytique ainsi que la valeur du gain AC
($v_{o+}-v_{o-}$)/($v_{i+}-v_{i-}$).
\begin{figure}
    \centering
    \includegraphics[width=10cm]{figures/fig11-2.png}
    \caption{Exercice 1}
    \label{fig11-2}
\end{figure}

\hspace{1cm}\hrule\hspace{1cm}

Comme précédemment, par symétrie du circuit, le courant dans chaque branche est
identique. Cela signifie que tous les NMOS ont la même transconductance $g_{mn}$
et la même résistance d'Early $r_{0n}$. De même, tous les PMOS ont la même 
transconductance $g_{mp}$ et la même résistance d'Early $r_{0p}$.
Le schéma petit signal du circuit s'obtient de la même façon que le précédent.
A nouveau, on néglige les résistances d'Early des transistors connectés en diodes
car $\frac{1}{g_{mp}} \ll r_{0p}$. On peut également négliger les résistances d'Early
des 2 NMOS centraux car, à nouveau, $\frac{1}{g_{mp}} \ll r_{0n}$.

\begin{center}
    \begin{circuitikz}
        \draw
        (0, 4) node[ground, yscale=-1] {}
        to[R=$1/(2g_{mp})$] (0, 2)
        	(0, 2) node[anchor=north] {$v_x$}
        (0, 0) node[ground] {}
        (0, 2) -- (2, 2) to[cI=$g_{mn} v_{i-}$] (2, 0) -- (0, 0)
        (0, 2) -- (-2, 2) to[cI=$g_{mn} v_{i+}$] (-2, 0) -- (0, 0)
        (0, 4) -- (4.5, 4) to[cI=$-g_{mp}v_x$] (4.5, 2) to[cI=$g_{mn} v_{i-}$] (4.5, 0) -- (2, 0)
        (0, 4) -- (-4.5, 4) to[cI=$-g_{mp}v_x$] (-4.5, 2) to[cI=$g_{mn} v_{i+}$] (-4.5, 0) -- (2, 0)
        (4.5, 4) -- (7, 4) to[R=$r_{0p}$] (7, 2) -- (4.5, 2)
        (-4.5, 4) -- (-7, 4) to[R=$r_{0p}$] (-7, 2) -- (-4.5, 2)
        (7, 2) to[R=$r_{0n}$] (7, 0) -- (4.5, 0)
        (-7, 2) to[R=$r_{0n}$] (-7, 0) -- (-4.5, 0)
        (7, 2) to[-o] (7.5, 2) node[anchor=west] {$v_{o+}$}
        (-7, 2) to[-o] (-7.5, 2) node[anchor=east] {$v_{o+}$}
        ;
    \end{circuitikz}
\end{center}

Ce schéma petit signal contient en fait 3 circuits indépendants. Le premier
nous permet de trouver la tension inconnue $v_x$
\[v_x = -\frac{g_{mn}}{2g_{mp}}(v_{i+}+v_{i-}).\]

\begin{center}
	\begin{circuitikz}
		\draw
		(2,0) -- (0, 0) to [cI_=$g_{mn}(v_{i+}+v_{i-})$] (0,-2) -- (2,-2)
		(2, 0) to[R=$\frac{1}{2g_{mp}}$] (2, -2)
		(1,-2) node[ground] {}
		(2,0) to[short,-o] (2.5,0) node[anchor=west] {$v_x$}
		;
 	\end{circuitikz}
\end{center}

Les deux autres nous permettent de trouver $v_{o+}$ et $v_{o-}$. Pour simplifier
les notations, on utilise $g_{dn}$ et $g_{dp}$ pour désigner les conductances
d'Early. Le circuit de gauche nous permet d'obtenir
\begin{align*}
	v_{o+} 	&= \frac{(g_{mn}v_{i+} + g_{mp}v_x)}{g_{dn} + g_{dp}} \\
			&= \frac{1}{2}\frac{g_{mn}}{g_{dn}+g_{dp}}(v_{i+}-v_{i-})
\end{align*}
tandis que le circuit de droite nous permet d'obtenir
\begin{align*}
	v_{o-} 	&= \frac{(g_{mn}v_{i-} + g_{mp}v_x)}{g_{dn} + g_{dp}} \\
			&= \frac{1}{2}\frac{g_{mn}}{g_{dn}+g_{dp}}(v_{i-}-v_{i+}).
\end{align*}

\begin{center}
	\begin{circuitikz}
		\draw
		(2,0) -- (0, 0) to [cI_=$g_{mn}v_{i+}+g_{mp}v_x$] (0,-2) -- (2,-2)
		(2, 0) to[R=$r_{0p}||r_{0n}$] (2, -2)
		(1,-2) node[ground] {}
		(2,0) to[short,-o] (2.5,0) node[anchor=west] {$v_{o+}$}
		;
 	\end{circuitikz}
 	\begin{circuitikz}
		\draw
		(2,0) -- (0, 0) to [cI_=$g_{mn}v_{i-}+g_{mp}v_x$] (0,-2) -- (2,-2)
		(2, 0) to[R=$r_{0p}||r_{0n}$] (2, -2)
		(1,-2) node[ground] {}
		(2,0) to[short,-o] (2.5,0) node[anchor=west] {$v_{o-}$}
		;
 	\end{circuitikz}
\end{center}

On a donc finalement
\[ \frac{v_{o+}-v_{o-}}{v_{i+}-v_{i-}} = \frac{g_{mn}}{g_{dn}+g_{dp}}.\]
Il ne reste plus qu'à calculer la valeur de $g_{mn}$, $g_{dn}$ et $g_{dp}$ en
s'aidant dans la table 9.1 du CMOS.
On sait que $g_{mn} = \frac{2I_D}{V_{OV}}$. Vu la symétrie du circuit, $I_D =
\SI{2.5}{\micro\ampere}$. On peut ensuite calculer
\[ V_{OV} = \sqrt{\frac{2I_D}{KP_n\frac{W}{L}}} = \SI{0.091}{\volt}\]
et donc $g_{mn} = \SI{54.8}{\micro\ampere\per\volt}$. Les conductances d'Early
sont quand à elle données par $g_{dn} = \lambda_n I_D = \SI{25}{\nano\siemens}$
et $g_{dp} = \lambda_pI_D = \SI{31.25}{\nano\siemens}$. Le gain vaut donc
$\SI{973.73}{\volt\per\volt}$.

\paragraph{Remarque}
On peut arriver au même résultat par superposition.

\end{document}

