TP1/ex3
1.
Le circuit a une relation linéaire entre le courant d'entrée et la tension de sortie:
\[V_{out} = \alpha + \beta I_{in}.\]
Cette relation sera démontrée dans les points suivants.

On peut cependant analyser intuitivement le comportement du circuit: on suppose que les transistors sont en régime de saturation. Dès lors, en partant d'un état donné, si la tension $V_{out}$ augmente, $V_{gs1}$ augmente et $V_{gs2}$ diminue, et donc $I_{ds1}$ augmente tandis que $I_{ds2}$ diminue. Par Kirchoff, $I_{in}$ augmente.
On peut appliquer le même raisonnement dans le cas où $V_{out}$ diminue. Par conséquent, $V_{out}$ suit les variations de $I_{in}$.

2.
Tout d'abord, on observe que le transistor Q1 est monté en diode. Il y a donc deux états possibles:
\begin{itemize}
\item bloqué si $V_{out} < V_{T0,n}$,
\item en saturation si $V_{out} > V_{T0,n}$.
\end{itemize}

Pour Q2:
\begin{itemize}
\item bloqué si $V_{bias} - V_{out} < V_{T0,n}$,
\item en saturation si $V_{bias} - V_{out} > V_{T0,n}$ et $V_{DD} - V_{out} > V_{bias} - V_{T0,n}$.
\item en triode si $V_{bias} - V_{out} > V_{T0,n}$ et $V_{DD} - V_{out} < V_{bias} - V_{T0,n}$.
\end{itemize}

Dans le cas où $V_{DD} < V_{bias} - V_{T0,n}$, on peut passer de ces conditions sur la tension $V_{out}$ à des conditions sur le courant $I_{in}$ en utilisant les équations des transistors lorsqu'ils sont en saturation (plage de fonctionnement "normal" du circuit).

\begin{align*}
    I_1 &= \frac{k}{2}\left(V_{out} - V_{T0,n}\right)^2 \\
    I_2 &= \frac{k}{2}\left(V_{bias} - V_{out} - V_{T0,n}\right)^2 \\
    I_{1} &= I_{in} + I_{2}
\end{align*}
    
En résolvant ces équations, on trouve
\[I_{in} = \frac{k}{2}\left(2 V_{T0,n} V_{bias} + (2 V_{bias} - 4 V_{T0,n}) V_{out} - V_{bias}^2\right)\]

En prenant en compte la valeur de $V_{T0,n}$ et $V_{bias} = \SI{4}{V}$, on trouve
\[I_{in} = k\left(\SI{2}{V} V_{out} - \SI{4}{V^2}\right)\]
et donc
\begin{itemize}
\item si $I_{in} < -\SI{2}{V^2}k = -\SI{78}{\micro\ampere}$, Q1 est bloqué,
\item si $I_{in} > \SI{2}{V^2}k = \SI{78}{\micro\ampere}$, Q2 est bloqué,
\item sinon, les deux transistors sont en saturation.
\end{itemize}

3.
Du point précédent, on trouve $-\SI{78}{\micro\ampere} < I_{in} < \SI{78}{\micro\ampere}$.

4.

Caractéristique:
\begin{center}
\includegraphics[width=0.6\textwidth,height=0.4\textwidth]{figures/ex3}
\end{center}

Equation de la caractéristique:
\begin{itemize}
    \item Si $I_{in} < \SI{-78}{\micro\ampere}$, Q1 est bloqué et
        \[I_{in} = - I_2 = -\frac{k}{2}\left(V_{bias} - V_{out} - V_{T0,n}\right)^2,\]
        donc \[V_{out} = V_{bias} - V_{T0,n} - \sqrt{\frac{2}{k}I_{in}}\]
    \item Si $I_{in} > \SI{78}{\micro\ampere}$, Q2 est bloqué et
        \[I_{in} = I_1 = \frac{k}{2}\left(V_{out} - V_{T0,n}\right)^2,\]
        donc \[V_{out} = V_{T0,n} + \sqrt{\frac{2}{k}I_{in}}\]
\end{itemize}


5.

On mesure la plage de fonctionnement linéaire par rapport à la sortie. La plage de fonctionnement linéaire s'étend entre $V_{T0,n}$ et $V_{bias} - V_{T0,n}$. On a donc une plage obtimale pour $V_{bias} = \SI{6}{V}$ (si $V_{bias} > \SI{6}{V}$, Q2 fonctionne en triode, et la relation entrée-sortie n'est plus linéaire).

6.

Si $k_1 \neq k_2$, on ne peut plus simplifier les termes en $V_{out}^2$ dans les équations (voir point 2), la relation entrée-sortie n'est donc plus linéaire.

%Enoncé:
%k: 10^-5, pas 10^5
%point 2: en fonction de V_out et V_bias, puis I_in ? 
%k = k1 = k2
%point 2: extremement calculatoire en symbolique si V_bias quelconque => reformuler / fixer V_bias ?
%I_{in} < 0 : peu intéressant.
