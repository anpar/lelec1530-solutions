\documentclass[frenchb,DIV=14]{scrartcl}

\usepackage[utf8x]{inputenc}
\usepackage[T1]{fontenc}
\usepackage{lmodern}
\usepackage[binary-units=true]{siunitx}
% Color
% cfr http://en.wikibooks.org/wiki/LaTeX/Colors
\usepackage{color}
\usepackage[usenames,dvipsnames,svgnames,table]{xcolor}
\definecolor{dkgreen}{rgb}{0.25,0.7,0.35}
\definecolor{dkred}{rgb}{0.7,0,0}
\usepackage{graphicx}
\usepackage{url}
\usepackage{tikz}
\usepackage{pgfplots}
\usepackage{microtype}
\usepackage{xspace}

\usepackage{hyperref}
\usepackage{todonotes}
\usepackage{epstopdf}

\usepackage{multirow}
\usepackage{tabularx} % tabular with automatic line-break
\newcolumntype{Y}{>{\centering\arraybackslash}X} % centered column

\newcommand{\matlab}{\textsc{Matlab}}

% Math symbols
\usepackage{amsmath}
\usepackage{amssymb}
\usepackage{amsthm}
\DeclareMathOperator*{\argmin}{arg\,min}
\DeclareMathOperator*{\argmax}{arg\,max}

% Unit vectors
\usepackage{esint}
\usepackage{esvect}
\newcommand{\kmath}{k}
\newcommand{\xunit}{\hat{\imath}}
\newcommand{\yunit}{\hat{\jmath}}
\newcommand{\zunit}{\hat{\kmath}}
\newcommand{\uunit}{\hat{\umath}}

% Elec
\newcommand{\B}{\vec B}
\newcommand{\E}{\vec E}
\newcommand{\EMF}{\mathcal{E}}
\newcommand{\perm}{\varepsilon} % permittivity

\newcommand{\bigoh}{\mathcal{O}}
\newcommand\eqdef{\triangleq}

\DeclareMathOperator{\newdiff}{d} % use \dif instead
\newcommand{\dif}{\newdiff\!}
\newcommand{\fpart}[2]{\frac{\partial #1}{\partial #2}}
\newcommand{\ffpart}[2]{\frac{\partial^2 #1}{\partial #2^2}}
\newcommand{\fdpart}[3]{\frac{\partial^2 #1}{\partial #2\partial #3}}
\newcommand{\fdif}[2]{\frac{\dif #1}{\dif #2}}
\newcommand{\ffdif}[2]{\frac{\dif^2 #1}{\dif #2^2}}
\newcommand{\constant}{\ensuremath{\mathrm{cst}}}
\newcommand{\norm}[1]{\left\lVert#1\right\rVert}

\usepackage{babel}
% Listing
% always put it after babel
% http://tex.stackexchange.com/questions/100717/code-in-lstlisting-breaks-document-compile-error
\usepackage{listings}

% Put caption after babel
\usepackage{caption}
\usepackage{subcaption}

\definecolor{mygreen}{rgb}{0,0.6,0}
\definecolor{mygray}{rgb}{0.5,0.5,0.5}
\definecolor{mymauve}{rgb}{0.58,0,0.82}
\lstset{ %
  language=Matlab,
  backgroundcolor=\color{white},   % choose the background color; you must add \usepackage{color} or \usepackage{xcolor}
  basicstyle=\footnotesize,        % the size of the fonts that are used for the code
  breakatwhitespace=false,         % sets if automatic breaks should only happen at whitespace
  breaklines=true,                 % sets automatic line breaking
  captionpos=b,                    % sets the caption-position to bottom
  commentstyle=\color{mygreen},    % comment style
  deletekeywords={...},            % if you want to delete keywords from the given language
  escapeinside={\%*}{*)},          % if you want to add LaTeX within your code
  extendedchars=true,              % lets you use non-ASCII characters; for 8-bits encodings only, does not work with UTF-8
  frame=single,	                   % adds a frame around the code
  keepspaces=true,                 % keeps spaces in text, useful for keeping indentation of code (possibly needs columns=flexible)
  keywordstyle=\color{blue},       % keyword style
  otherkeywords={*,...},           % if you want to add more keywords to the set
  numbers=left,                    % where to put the line-numbers; possible values are (none, left, right)
  numbersep=5pt,                   % how far the line-numbers are from the code
  numberstyle=\tiny\color{mygray}, % the style that is used for the line-numbers
  rulecolor=\color{black},         % if not set, the frame-color may be changed on line-breaks within not-black text (e.g. comments (green here))
  showspaces=false,                % show spaces everywhere adding particular underscores; it overrides 'showstringspaces'
  showstringspaces=false,          % underline spaces within strings only
  showtabs=false,                  % show tabs within strings adding particular underscores
  stepnumber=1,                    % the step between two line-numbers. If it's 1, each line will be numbered
  stringstyle=\color{mymauve},     % string literal style
  tabsize=2,	                   % sets default tabsize to 2 spaces
  title=\lstname                   % show the filename of files included with \lstinputlisting; also try caption instead of title
}

\KOMAoptions{DIV=last}




\titlehead{}
\subject{LELEC1530}
\title{Séance 7 - Logique digitale de base}
\subtitle{Solutions}
\author{\small Gaëtan \textsc{Cassiers} \and\small Antoine \textsc{Paris}}
\date{}

\begin{document}
\maketitle

\section*{Exercice 1 : modèles digitaux}
\paragraph{NMOS}
Si initialement $V_{IN} = \SI{0}{\volt}$, on suppose que la capacité de charge,
$C_L$, est chargée à $V_{DD}$ (sinon, pas de transition à la sortie quand $V_{IN}$
passe à $V_{DD}$). En utilisant le modèle digital simple du MOSFET avec les
capacités parasites (section 10.1 du Baker), on obtient le circuit suivant.

\begin{center}
	\begin{tikzpicture}
		\draw
    	%(0, 2) node[anchor=east] {$V_{IN}$} to [short, o-] (1, 2)
    	%(1, 2) to [C=$\frac{3}{2}C_{ox}$] (1, 0)
    	%(1, 0) -- (4, 0)
    	(4, 0) to [R=$R_n$] (4, 2)
    	(4, 2) to [switch] (6,2)
    	(6, 2) to [C=$C_{ox}$] (6,0) -- (4, 0)
    	(6, 2) -- (8, 2) to [C=$C_L$] (8,0) -- (6, 0)
    	(8, 2) to [short, -o] (9,2) node[anchor=west] {$V_{OUT}$}
    	(6, 0) node[ground] {};
	\end{tikzpicture}
\end{center}

Toujours selon ce même modèle digital simple du MOSFET, le switch se ferme
quand $V_{GS} = V_{IN} > \frac{V_{DD}}{2}$.
Soit $C_{tot} = C_L + C_{ox}$ la mise en parallèle de $C_L$ et $C_{ox}$. Le
circuit, une fois le switch fermé, se résume à un simple circuit RC.

\begin{center}
	\begin{tikzpicture}
		\draw
    	(0, 0) to [R=$R_n$] (0, 2)
    	(0, 0) -- (2, 0)
    	(2, 0) to [C=$C_{tot}$] (2, 2) -- (0, 2)
    	(2, 2) to [short, -o] (3,2) node[anchor=west] {$V_{OUT}$}
    	(1, 0) node[ground] {};
	\end{tikzpicture}
\end{center}

La tension $V_{OUT}$ évolue selon l'équation différentielle
\[ \fdif{V_{OUT}}{t} + \frac{V_{OUT}}{R_nC_{tot}} = 0 \]
dont la solution est une exponentielle de constante de temps $\tau = R_nC_{tot}$
\[ V_{OUT}(t) = V_{DD}\exp\left(-t/\tau\right).\]
Soit $t_{PHL}$ le délai de propagation à 50\% de $V_{DD}$ (high) à \SI{0}{\volt} (low)
\footnote{Ce délai et les autres sont définit dans la section 10.1.3 du Baker.}.
$t_{PHL}$ est donné par
\[ V_{OUT}(t_{PHL}) = V_{DD}\exp\left(-t_{PHL}/\tau\right) = \frac{V_{DD}}{2} \]
et donc
\[ t_{PHL} = \ln(2)\cdot R_n\cdot C_{tot} \approx 0.7\cdot R_n\cdot C_{tot}. \]
\`{A} l'avenir, on utilisera directement cette approximation sans repasser par le calcul
complet (mais faire une fois ce calcul permet de voir d'où elle vient).

La résistance $R'_n$ donnée dans l'énoncé doit encore être divisée par $W$ pour
obtenir $R_n$
\[ R_n = \frac{R'_n}{W}. \]
Cette formule n'est cependant valable que pour $L=1$ (notons que $L=1$ veut
dire que la longueur du transistor est de \SI{50}{\nano\meter}, comme le
scale factor est de \SI{50}{\nano\meter}). Ici, la longueur du transistor
est de $\SI{1}{\micro\meter}$, soit $L=20$, on a donc finalement
\[ R_n = \SI{34}{\kilo\ohm\micro\meter}\frac{20}{\SI{10}{\micro\meter}} =
\SI{68}{\kilo\ohm}. \]
Ensuite, pour obtenir $C_{ox}$, il faut multiplier $C'_{ox}$ par la surface du
transistor $WL$
\[ C_{ox} = C'_{ox}\cdot(WL) = \SI{625}{\atto\farad}. \]
Comme $C_L$ n'est pas donnée dans l'énoncé, on suppose $C_L = \SI{50}{\femto\farad}$.
Enfin, le délai est donné par
\[ t_{PHL} = \SI{2.41}{\nano\second}. \]

\paragraph{PMOS}
Pour le PMOS, le raisonnement est exactement identique. On suppose cette fois que la
capacité de charge $C_L$ est initialement chargée à \SI{0}{\volt}. Le modèle digital
simple du MOSFET appliqué au PMOS nous donne le circuit suivant.

\begin{center}
	\begin{tikzpicture}
		\draw
    	(0, 4) node[vcc] {$V_{DD}$} to [R=$R_p$] (0, 2)
    	(0, 2) to [switch] (3, 2)
    	(3, 2) to [C=$C_{ox}$] (3, 4) node[vcc] {$V_{DD}$}
    	(3, 2) -- (4, 2) to [C=$C_L$] (4,0) node[ground] {}
    	(4, 2) to [short, -o] (5, 2) node[anchor=west] {$V_{OUT}$};
	\end{tikzpicture}
\end{center}

Le switch se ferme cette fois dès que $V_{SG} = V_{DD} - V_{IN} > \frac{V_{DD}}{2}$.
Une fois ce switch fermé, le circuit se résume à nouveau à un simple circuit RC.

\begin{center}
	\begin{tikzpicture}
		\draw
    	(0, 4) node[vcc] {$V_{DD}$} to [R=$R_p$] (0, 2)
    	(0, 2) -- (3, 2)
    	(3, 2) to [C=$C_{ox}$] (3, 4) node[vcc] {$V_{DD}$}
    	(3, 2) -- (4, 2) to [C=$C_L$] (4,0) node[ground] {}
    	(4, 2) to [short, -o] (5, 2) node[anchor=west] {$V_{OUT}$};
	\end{tikzpicture}
\end{center}

Soit $C_{tot} = C_L + C_{ox}$. L'équation différentielle régissant la variation
de $V_{OUT}$ est donnée par
\[ \fdif{V_{OUT}}{t} + \frac{V_{OUT}}{R_pC_{tot}} = \frac{V_{DD}}{R_pC_{tot}} \]
et donc, en notant $\tau = R_nC_{tot}$,
\[ V_{OUT}(t) = V_{DD}\left(1 - \exp\left(-t/\tau\right)\right). \]
Le délai $t_{PLH}$ est donné par
\[ V_{OUT}(t_{PLH}) = V_{DD}\left(1 - \exp\left(-t_{PLH}/\tau\right)\right) = \frac{V_{DD}}{2} \]
et donc
\[ t_{PLH} = \ln(2)\cdot R_p\cdot C_{tot} \approx 0.7\cdot R_p\cdot C_{tot}. \]
De la même manière que pour le NMOS,
\[ R_p = \SI{68}{\kilo\ohm\micro\meter}\frac{20}{\SI{20}{\micro\meter}} =
\SI{68}{\kilo\ohm}. \]
Ensuite,
\[ C_{ox} = C'_{ox}\cdot(WL) = \SI{1.25}{\femto\farad}. \]
Finalement,
\[ t_{PLH} = \SI{2.44}{\nano\second}. \]

\paragraph{Remarque} On voit donc que dimensionner le PMOS en choisissant une
largeur deux fois plus grande que pour le NMOS permet d'avoir $t_{PHL} \approx
t_{PLH}$ (en négligeant les capacités parasites par rapport à la capacité
de charge, on a même $t_{PHL} = t_{PLH}$). Cependant, cela a aussi pour effet
d'augmenter la capacité d'entrée $C_{in} = \frac{3}{2}C_{ox}$ du circuit avec
le PMOS.

\section*{Exercice 2: inverseur CMOS}

\subsection*{Question 1: Analyse DC}

\paragraph{Calcul des paramètres}
\begin{align*}
    k_n &= \left(\frac{W}{L}\right)_n \mu_n C_{ox} = \SI{200}{\micro A/V^2} \\
    k_p &= \left(\frac{W}{L}\right)_p \mu_p C_{ox} = \SI{200}{\micro A/V^2} \\
\end{align*}

\og Transition infiniment lente\fg signifie que l'on analyse ce qu'il se passe
en DC, pour différentes valeurs de l'entrée (ce que fait SPICE pour une analyse
\texttt{.DC}).

On est en DC, en grand signal, le courant de sortie est donc nul (la sortie est
connectée à des capacités). Les courants dans les deux transistors sont donc
égaux.

\paragraph{Pour $V_{IN} < V_{t0,n}$:}
M1 est bloqué et M2 est passant.
Comme $I_{2} = 0$, le M2 ne peut pas être en saturation,
donc il est en triode:
\[I_{OUT} = k_p \left(V_{OV,M2} - \frac{1}{2} V_{SD,M2}\right)V_{SD,M2} = \SI{0}{A}\]
On a donc $V_{SD,M2} = \SI{0}{V}$, et donc $V_{OUT} = V_{DD}$.

\paragraph{Pour $V_{IN} > V_{DD} - \left|V_{t0,p}\right|$:}
M2 est bloqué et M1 passant. Par un raisonnement similaire à ci-dessus,
on trouve $V_{OUT} = \SI{0}{V}$.

\paragraph{Pour $V_{t_0,n} < V_{IN} < V_{DD} - \left|V_{t0,p}\right|$:}
Les deux transistors sont passants.

En partant de $V_{IN} = V_{DD} - \left|V_{t0,p}\right|$,
si on diminue $V_{IN}$, M1 reste en triode et M2 devient passant
(en saturation car $V_{SD} \approx V_{DD}$, alors que $V_{OV} \approx \SI{0}{V}$).

On a donc
\begin{align*}
    I_{M1} &= k_n \left(V_{OV,M1}-\frac{1}{2}V_{DS,M1}\right)V_{DS,M1} \\
    I_{M2} &= \frac{k_p}{2} V_{OV,M2}^2 \\
    I_{M1} &= I_{M2} \\
\end{align*}

On peut donc résoudre ce système pour trouver $V_{OUT}$ en fonction
de $V_{IN}$ (en utilisant $k_n = k_p$ et $V_t = V_{t0,n} = |V_{t0,p}|$:
\[V_{OUT} = V_{IN}-V_t - \sqrt{(V_{IN}-V_t)^2-(V_{DD}-V_{IN}-V_t)^2}.\]

La limite de ce régime de fonctionnement est atteinte quand
$V_{OV,M2} = V_{SD,M2}$, donc $\frac{k_p}{2}V_{OV,M2}^2 = \frac{k_n}{2}V_{OV,M1}^2$.
Comme $k_n = k_p$, cela donne $V_{IN} = V_{DD}/2$.

Si on diminue encore $V_{IN}$, on rentre dans une zone où les deux transistors sont en saturation.
Sans connaitre les paramètres de l'effet Early, on ne peut pas calculer
$V_{OUT}$ en fonction de $V_{IN}$ (si on néglige l'effet Early, on
obtient une caractéristique $V_{OUT}(V_{IN})$ verticale pour cette région,
ce qui n'a pas pas de sens physique).

Quand $V_{IN}$ est suffisament petit, on arrive dans une région où
M1 est en saturation et M2 en triode.
Les équations sont alors
\begin{align*}
    I_{M1} &= \frac{k_n}{2} (V_{IN} - V_{t0,n})^2 \\
    I_{M2} &= k_p \left(V_{OV,M2}-\frac{1}{2}V_{SD,M2}\right)V_{SD,M2} \\
    I_{M1} &= I_{M2} \\
\end{align*}

On peut alors tracer la courbe DC de l'inverseur CMOS:
\begin{center}
\includegraphics[width=0.7\textwidth,height=0.5\textwidth]{figures/ex2_1.tikz}
\end{center}

\subsection*{Question 2: Délai de propagation: calcul complet}

Comme $v_{IN} = V_{DD}$ sur toute la période considérée, M2 est tout le temps
bloqué. Pour M1, $v_{OV} = v_{IN} - V_{t0,n} = \SI{1.3}{V}$.
Comme on s'intéresse au délai à 50\%, $V_{DD}/2 < v_{OUT} < V_{DD}$.

M1 est donc d'abord en saturation
($v_{OUT}: V_{DD} \rightarrow v_{OV}$), puis en triode
($v_{OUT}: v_{OV} \rightarrow V_{DD}/2$).

\paragraph{Saturation}

Les équations du circuit sont
\begin{align*}
    i_{M1} &= \frac{k_n}{2}(v_{IN} - V_{t0,n})^2 \\
    i_C &= C_{rout}\fpart{v_{OUT}}{t} \\
    i_C &= -i_{M1} \\
\end{align*}

On a donc $\Delta v_{OUT} = v_{OV} - V_{DD} = \SI{-0.5}{V}$ et
$i_{M1} = \SI{169}{\micro\ampere}$.
Dès lors, $t_{SAT} = C_{rout}\frac{\Delta v_{OUT}}{-i_{M1}} = \SI{29.58}{\pico\second}$.

\paragraph{Triode}

Les équations du circuit sont alors
\begin{align*}
    i_{M1} &= k_n (v_{OV} - \frac{1}{2}v_{OUT})v_{OUT} \\
    i_C &= C_{rout}\fpart{v_{OUT}}{t} \\
    i_C &= -i_{M1} \\
\end{align*}

On a donc
\[C_{rout}\fpart{v_{OUT}}{t} = - k_n (v_{OV} - \frac{1}{2}v_{OUT})v_{OUT}.\]
Il faut résoudre cette équation différentielle (EDO du premier ordre) pour
la condition initiale $v_{OUT}(t_{SAT}) = V_{OV}$, et trouver ensuite
$t^* > t_{SAT}$ tel que $v_{OUT}(t^*) = V_{DD}/2$.

Pour simplifier les calculs, on va remplacer le terme quadratique
de l'équation par sont développement de Taylor au premier ordre.
Comme on résout l'équation pour $v_{OUT}\in [\SI{0.5}{V}, \SI{1.3}{V}]$,
on va faire le développement autour de \SI{0.9}{V}:
\[v_{OUT}^2 \approx 0.9^2 + 2\cdot 0.9(v_{OUT}-0.9)
= -0.9^2 + 2\cdot 0.9\cdot v_{OUT}.\]
Cela donne:
\[\fpart{v_{OUT}}{t} = - \frac{k_n}{C_{rout}} (v_{OV}-0.9) v_{OUT} - \frac{0.9^2}{2}\frac{k_n}{C_{rout}}.\]
La solution est donc du type
\[v_{OUT}(t) = \tau\left(\beta - \alpha\exp^{-t/\tau}\right)\]
avec $\tau = C_{rout} / k_n / (v_{OV}-0.9)$, $\beta = -0.9^2/2\cdot k_n/C_{rout}$ et $\alpha$ tel que
\[v_{OUT}(t_{SAT}) =  v_{OV}.\]
On trouve donc
\begin{align*}
    \tau &= \SI{125}{\pico\second} \\
    \alpha &= \left(\beta-\frac{v_{OV}}{\tau}\right)\exp^{t_{SAT}/\tau} = \SI{-2.34e10}{V/s} \\
    t^* &= -\tau\ln\left(\frac{\frac{V_{DD}/2}{\tau}-\beta}{\alpha}\right) = \SI{53.3}{ps} \\
\end{align*}

\subparagraph{Sans simplification} On peut aussi intégrer l'EDO numériquement,
ce qui donne $t^* = \SI{54.1}{ps}$.
\begin{center}
    \includegraphics[width=0.7\textwidth,height=0.4\textwidth]{figures/delai.tikz}
\end{center}
On constate que l'approximation était assez bonne (2\% d'erreur sur $t^*$).

\subsection*{Question 3: Délai de propagation: simplification}

Si on remplace le transistor par $R_{eq}$, on a l'équation
\[v_{OUT}(t) = V_{DD} \exp^{-t/(R_{eq}C_{rout})},\]
donc \[\frac{t^*}{R_{eq}C_{rout}} = \ln 2.\]
On trouve $R_{eq} = \SI{7.69}{\kilo\ohm}$.

Dans la zone linéaire (approximation linéaire du régime triode),
on avait $\tau = C_{rout} / (k_n v_{OV})$. Par conséquent,
$R_{lin} = 1 / (k_n v_{OV}) = \SI{3.85}{\kilo\ohm}$.

Finalement,
\[k = \frac{R_{eq}}{R_{lin}} = 2.00.\]

\subsection*{Question 4: Scaling de la technologie CMOS}

On suppose que la valeur calculée pour $k$ ne dépend pas de la technologie:
\[R_{eq} = k R_{lin} = \frac{k}{k_n (V_{DD} - V_t)} = \SI{7.14}{\kilo\ohm} \]
\[t_{PHL} = \SI{49.5}{\pico\second} \]

\section*{Exercice 3 : logique de passage}
\subsection*{Niveaux logiques}
\paragraph{MUX1: transmission gate} 
Ce multiplexeur est composé de deux transmission gates (TG).
Une TG combine un transistor NMOS et un transistor PMOS de façon à se débarasser
de leurs défauts lorsque ces transistors sont utilisés individuellement dans une
simple pass gate (PG). En effet, on sait qu'un NMOS passe bien un "0" mais ne passe
pas bien un "1". Au contraire, un PMOS passe bien un "1" mais ne passe pas bien un "0".
Dans une TG, les "1" et "0" passent tous les deux bien grâce à l'association PMOS/NMOS.
On a donc
\begin{align*}
	V_{OL} &= \SI{0}{\volt} & V_{OH} &= V_{DD}.
\end{align*}

\paragraph{MUX2: pass gate} 
\`{A} l'inverse du MUX1, MUX2 utilise de simple pass gates NMOS.
Comme expliqué précédemment, un NMOS ne passe pas bien un "1". Ici, on a donc
\begin{align*}
	V_{OL} &= \SI{0}{\volt} & V_{OH} &= V_{DD}-V_{t0,n}.
\end{align*}

\paragraph{MUX3: static CMOS gate}
Ici, on constate que les NMOS ne font passer que des "0" et que les PMOS ne
font passer que des "1". On se retrouve donc à nouveau avec
\begin{align*}
	V_{OL} &= \SI{0}{\volt} & V_{OH} &= V_{DD}.
\end{align*}

\subsection*{Temps de décharge}
Voir l'exercice 2.4.
\[R_{eq} = k R_{lin} = \frac{k}{k_n (V_{DD} - V_t)} = \SI{9.62}{\kilo\ohm} \]
\[t_{PHL} = \SI{66.6}{\pico\second} \]

\end{document}

