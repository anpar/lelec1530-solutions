\documentclass[frenchb,DIV=14]{scrartcl}

\usepackage[utf8x]{inputenc}
\usepackage[T1]{fontenc}
\usepackage{lmodern}
\usepackage{microtype}
\usepackage{xspace}
\usepackage[binary-units=true]{siunitx}
%\usepackage{color}
%\usepackage[usenames,dvipsnames,svgnames,table]{xcolor}
\usepackage{graphicx}
\usepackage{hyperref}
\usepackage{todonotes}
\usepackage{epstopdf}
\usepackage{array}
\usepackage{multicol}
\usepackage{multirow}
\usepackage{tabularx} % tabular with automatic line-break
\newcolumntype{Y}{>{\centering\arraybackslash}X} % centered column
\usepackage{ifthen}
\usepackage{amsmath}
\usepackage{grffile} % better name handling with graphicx
\usepackage{currfile} % provides relative file inclusion for tikzscale

\usepackage{tikz}
\usepackage{pgfplots}
\usepackage{tikz-3dplot}
\usepackage{pgfplots}
\usepackage{tikzscale}
\pgfplotsset{compat=newest}
\usetikzlibrary{plotmarks}
\usepackage{rotating}

\usepackage[french]{babel}
% Listing
% always put it after babel
% http://tex.stackexchange.com/questions/100717/code-in-lstlisting-breaks-document-compile-error
\usepackage{listings}

\definecolor{mygreen}{rgb}{0,0.6,0}
\definecolor{mygray}{rgb}{0.5,0.5,0.5}
\definecolor{mymauve}{rgb}{0.58,0,0.82}
\lstset{ %
  language=Matlab,
  backgroundcolor=\color{white},   % choose the background color; you must add \usepackage{color} or \usepackage{xcolor}
  basicstyle=\footnotesize,        % the size of the fonts that are used for the code
  breakatwhitespace=false,         % sets if automatic breaks should only happen at whitespace
  breaklines=true,                 % sets automatic line breaking
  captionpos=b,                    % sets the caption-position to bottom
  commentstyle=\color{mygreen},    % comment style
  deletekeywords={...},            % if you want to delete keywords from the given language
  escapeinside={\%*}{*)},          % if you want to add LaTeX within your code
  extendedchars=true,              % lets you use non-ASCII characters; for 8-bits encodings only, does not work with UTF-8
  frame=single,	                   % adds a frame around the code
  keepspaces=true,                 % keeps spaces in text, useful for keeping indentation of code (possibly needs columns=flexible)
  keywordstyle=\color{blue},       % keyword style
  otherkeywords={*,...},           % if you want to add more keywords to the set
  numbers=none,                    % where to put the line-numbers; possible values are (none, left, right)
  numbersep=5pt,                   % how far the line-numbers are from the code
  numberstyle=\tiny\color{mygray}, % the style that is used for the line-numbers
  rulecolor=\color{black},         % if not set, the frame-color may be changed on line-breaks within not-black text (e.g. comments (green here))
  showspaces=false,                % show spaces everywhere adding particular underscores; it overrides 'showstringspaces'
  showstringspaces=false,          % underline spaces within strings only
  showtabs=false,                  % show tabs within strings adding particular underscores
  stepnumber=2,                    % the step between two line-numbers. If it's 1, each line will be numbered
  stringstyle=\color{mymauve},     % string literal style
  tabsize=2,	                   % sets default tabsize to 2 spaces
  title=\lstname                   % show the filename of files included with \lstinputlisting; also try caption instead of title
}

\newcommand{\matlab}{\textsc{Matlab}}

% Math symbols
\usepackage{amsmath}
\usepackage{amssymb}
\usepackage{amsthm}
\DeclareMathOperator*{\argmin}{arg\,min}
\DeclareMathOperator*{\argmax}{arg\,max}


% Sets
\newcommand{\Z}{\mathbb{Z}}
\newcommand{\R}{\mathbb{R}}
\newcommand{\Rn}{\R^n}
\newcommand{\Rnn}{\R^{n \times n}}
\newcommand{\C}{\mathbb{C}}
\newcommand{\K}{\mathbb{K}}
\newcommand{\Kn}{\K^n}
\newcommand{\Knn}{\K^{n \times n}}

% Unit vectors
\usepackage{esint}
\usepackage{esvect}
\newcommand{\kmath}{k}
\newcommand{\xunit}{\hat{\imath}}
\newcommand{\yunit}{\hat{\jmath}}
\newcommand{\zunit}{\hat{\kmath}}
\newcommand{\uunit}{\hat{\umath}}

% rot & div & grad & lap
\DeclareMathOperator{\newdiv}{div}
\newcommand{\divn}[1]{\nabla \cdot #1}
\newcommand{\rotn}[1]{\nabla \times #1}
\newcommand{\grad}[1]{\nabla #1}
\newcommand{\gradn}[1]{\nabla #1}
\newcommand{\lap}[1]{\nabla^2 #1}


% Elec
\newcommand{\B}{\vec B}
\newcommand{\E}{\vec E}
\newcommand{\EMF}{\mathcal{E}}
\newcommand{\perm}{\varepsilon} % permittivity

\newcommand{\bigoh}{\mathcal{O}}
\newcommand\eqdef{\triangleq}

\DeclareMathOperator{\newdiff}{d} % use \dif instead
\newcommand{\dif}{\newdiff\!}
\newcommand{\fpart}[2]{\frac{\partial #1}{\partial #2}}
\newcommand{\ffpart}[2]{\frac{\partial^2 #1}{\partial #2^2}}
\newcommand{\fdpart}[3]{\frac{\partial^2 #1}{\partial #2\partial #3}}
\newcommand{\fdif}[2]{\frac{\dif #1}{\dif #2}}
\newcommand{\ffdif}[2]{\frac{\dif^2 #1}{\dif #2^2}}
\newcommand{\constant}{\ensuremath{\mathrm{cst}}}

%\KOMAoptions{DIV=last}

\usepackage{circuitikz}
\usetikzlibrary{babel}
\ctikzset{tripoles/mos style/arrows}

\makeatletter
\pgfcircdeclaremos{pmos}{
          \anchor{S}{
            \northeast
          }
          \anchor{source}{
            \northeast
          }
          \anchor{D}{
            \northeast
            \pgf@y=-\pgf@y
          }
          \anchor{drain}{
            \northeast
            \pgf@y=-\pgf@y
          }
}{%
            \pgfpathmoveto{\pgfpoint{\pgf@circ@res@right}{\pgf@circ@res@up}}
            \pgfpathlineto{\pgfpoint{\pgf@circ@res@right}
                {\pgfkeysvalueof{/tikz/circuitikz/tripoles/pmos/gate height}\pgf@circ@res@up}}
            \pgfpathlineto{\pgfpoint
                {\pgfkeysvalueof{/tikz/circuitikz/tripoles/pmos/base width}\pgf@circ@res@left}
                {\pgfkeysvalueof{/tikz/circuitikz/tripoles/pmos/gate height}\pgf@circ@res@up}}
            \pgfusepath{draw}

        \ifpgf@circuit@mos@arrows
            \pgfscope             
            \pgfslopedattimetrue 
            \pgfallowupsidedownattimetrue
            \pgfresetnontranslationattimefalse
            \pgftransformlineattime{.4}{%
                \pgfpoint%
                    {\pgf@circ@res@right}%
                    {\pgfkeysvalueof{/tikz/circuitikz/tripoles/pmos/gate height}\pgf@circ@res@up}%
            }{%
                \pgfpoint
                    {\pgfkeysvalueof{/tikz/circuitikz/tripoles/pmos/gate width}\pgf@circ@res@left}%
                    {\pgfkeysvalueof{/tikz/circuitikz/tripoles/pmos/gate height}\pgf@circ@res@up}%
            }
            \pgfnode{currarrow}{center}{}{}{\pgfusepath{stroke}}
            \endpgfscope
        \fi

            \pgfscope
            \pgfpathmoveto{\pgfpoint
                {\pgfkeysvalueof{/tikz/circuitikz/tripoles/pmos/base width}\pgf@circ@res@left}
                {\pgfkeysvalueof{/tikz/circuitikz/tripoles/pmos/base height}\pgf@circ@res@up}}
            \pgfpathlineto{\pgfpoint
                {\pgfkeysvalueof{/tikz/circuitikz/tripoles/pmos/base width}\pgf@circ@res@left}
                {\pgfkeysvalueof{/tikz/circuitikz/tripoles/pmos/base height}\pgf@circ@res@down}}
            \pgfsetlinewidth{2\pgflinewidth}
            \pgfusepath{draw}
            \endpgfscope

            \pgfpathmoveto{\pgfpoint
                {\pgfkeysvalueof{/tikz/circuitikz/tripoles/pmos/base width}\pgf@circ@res@left}
                {\pgfkeysvalueof{/tikz/circuitikz/tripoles/pmos/gate height}\pgf@circ@res@down}}
            \pgfpathlineto{\pgfpoint{\pgf@circ@res@right}
                {\pgfkeysvalueof{/tikz/circuitikz/tripoles/pmos/gate height}\pgf@circ@res@down}}
            \pgfpathlineto{\pgfpoint{\pgf@circ@res@right}{\pgf@circ@res@down}}      

            \pgfpathmoveto{\pgfpoint
                {\pgfkeysvalueof{/tikz/circuitikz/tripoles/pmos/gate width}\pgf@circ@res@left}
                {\pgfkeysvalueof{/tikz/circuitikz/tripoles/pmos/gate height}\pgf@circ@res@up}}
            \pgfpathlineto{\pgfpoint
                {\pgfkeysvalueof{/tikz/circuitikz/tripoles/pmos/gate width}\pgf@circ@res@left}
                {\pgfkeysvalueof{/tikz/circuitikz/tripoles/pmos/gate height}\pgf@circ@res@down}}


            \pgfpathmoveto{\pgfpoint
                {\pgfkeysvalueof{/tikz/circuitikz/tripoles/pmos/gate width}\pgf@circ@res@left}
                {\pgf@circ@res@up+\pgf@circ@res@down}}
            \pgfpathlineto{\pgfpoint{\pgf@circ@res@left}{\pgf@circ@res@up+\pgf@circ@res@down}}
            \pgfusepath{draw}

%           \pgfpathcircle{\pgfpoint
%               {\pgfkeysvalueof{/tikz/circuitikz/tripoles/pmos/gate width}\pgf@circ@res@left - \pgfkeysvalueof{/tikz/circuitikz/nodes width}*\pgfkeysvalueof{/tikz/circuitikz/bipoles/length}}
%               {\pgf@circ@res@up+\pgf@circ@res@down}}{\pgfkeysvalueof{/tikz/circuitikz/nodes width}*\pgfkeysvalueof{/tikz/circuitikz/bipoles/length}}
%           \pgfusepath{draw,fill}      

}

\titlehead{}
\subject{LELEC1530}
\title{Séance 7 - Logique digitale de base}
\subtitle{Solutions}
\author{\small Gaëtan \textsc{Cassiers} \and\small Antoine \textsc{Paris}}
\date{}

\begin{document}
\maketitle

\section*{Exercice 1}

\section*{Exercice 2}

\subsection*{Question 1: Analyse DC}

\paragraph{Calcul des paramètres}
\begin{align*}
    k_n &= \left(\frac{W}{L}\right)_n \mu_n C_{ox} L_{min}^2 = TODO \\
    k_p &= \left(\frac{W}{L}\right)_p \mu_p C_{ox} L_{min}^2 = TODO \\
\end{align*}

\og Transition infiniment lente\fg signifie que l'on analyse ce qu'il se passe
en DC, pour différentes valeurs de l'entrée (ce que fait SPICE pour une analyse
\texttt{.DC}).

On est en DC, en grand signal, le courant de sortie est donc nul (la sortie est
connectée à des capacités).

\paragraph{$V_{IN} < V_{t0,n}$}
M1 est bloqué et M2 est passant.
Comme $I_{OUT} \neq 0$, le M2 ne peut pas être en saturation,
donc il est en triode:
\[I_{OUT} = k_p \left(V_{OV,M2} - \frac{1}{2} V_{SD,M2}\right)V_{SD,M2} = \SI{0}{A}\]
On a donc $V_{SD,M2} = \SI{0}{V}$, et donc $V_{OUT} = V_{DD}$.

\paragraph{$V_{IN} > V_{DD} - \left|V_{t0,p}\right|$}
M2 est bloqué et M1 passant. Par un raisonnement similaire à ci-dessus,
on trouve $V_{OUT} = \SI{0}{V}$.

\paragraph{$V_{t_0,n} < V_{IN} < V_{DD} - \left|V_{t0,p}\right|$}
Les deux transistors sont passants.

En partant de $V_{IN} = V_{DD} - \left|V_{t0,p}\right|$,
si on diminue $V_{IN}$, M1 reste en triode et M2 devient passant
(en saturation car $V_{SD} \approx V_{DD}$, alors que $V_{OV} \approx \SI{0}{V}$).

On a donc
\begin{align*}
    I_{M1} &= \frac{k_n}{2} (V_{IN} - V_{t0,n})^2 \\
    I_{M2} &= k_p \left(V_{OV,M2}-\frac{1}{2}V_{SD,M2}\right)V_{SD,M2} \\
    I_{M1} &= I_{M2} \\
\end{align*}

On peut donc résoudre ce système pour trouver $V_{OUT}$ en fonction
de $V_{IN}$.

La limite de ce régime de fonctionnement est atteinte quand
$V_{OV,M2} = V_{SD,M2}$, donc $\frac{k_p}{2}V_{OV,M2}^2 = \frac{k_n}{2}V_{OV,M1}^2$.

Au-delà, on rentre dans une zone où les deux transistors sont en saturation.
Sans connaitre les paramètres de l'effet Early, on ne peut pas calculer
$V_{OUT}$ en fonction de $V_{IN}$ (si on néglige l'effet Early, on
obtient une caractéristique $V_{OUT}(V_{IN})$ verticale pour cette région,
ce qui n'a pas pas de sens physique).

Quand $V_{IN}$ est suffisament grand, on arrive dans une région où
M1 est en saturation et M2 en triode.
Les équations sont alors
\begin{align*}
    I_{M1} &= k_n \left(V_{OV,M1}-\frac{1}{2}V_{DS,M1}\right)V_{DS,M1} \\
    I_{M2} &= \frac{k_p}{2} V_{OV,M2}^2 \\
    I_{M1} &= I_{M2} \\
\end{align*}

\todo{Ajouter courbe $V_out(vin))$}

\subsection*{Question 2: Délai de propagation: calcul complet}

Comme $v_{IN} = V_{DD}$ sur toute la période considérée, M2 est tout le temps
bloqué. Pour M1, $v_{OV} = v_{IN} - V_{t0,n} = \SI{1.3}{V}$.
Comme on s'intéresse au délai à 50\%, $V_{DD}/2 < v_{OUT} < V_{DD}$.

M1 est donc d'abord en saturation
($v_{OUT}: V_{DD} \rightarrow v_{OV}$), puis en triode
($v_{OUT}: v_{OV} \rightarrow V_DD/2$).

\paragraph{Saturation}

Les équations du circuit sont
\begin{align*}
    i_{M1} &= \frac{k_n}{2}(v_{IN} - V_{t0,n})^2 \\
    i_C &= C_{rout}\fpart{v_{OUT}}{t} \\
    i_C &= -i_{M1} \\
\end{align*}

On a donc $\Delta v_{OUT} = v_{OV} - V_{DD} = \SI{-0.5}{V}$ et
$i_{M1} = TODO$.
Dès lors, $t_{SAT} = C_{rout}\frac{\Delta v_{OUT}}{-i_{M1}} = TODO$.

\paragraph{Triode}

Les équations du circuit sont alors
\begin{align*}
    i_{M1} &= k_n (v_{OV} - \frac{1}{2}v_{OUT})v_{OUT} \\
    i_C &= C_{rout}\fpart{v_{OUT}}{t} \\
    i_C &= -i_{M1} \\
\end{align*}

On a donc
\[C_{rout}\fpart{v_{OUT}}{t} = - k_n (v_{OV} - \frac{1}{2}v_{OUT})v_{OUT}.\]
Il faut résoudre cette équation différentielle (EDO du premier ordre) pour
la condition initiale $v_{OUT}(t_{SAT}) = V_{OV}$, et trouver ensuite
$t^* > t_{SAT}$ tel que $v_{OUT}(t^*) = V_{DD}/2$.

Pour simplifier les calculs, on va négliger le terme quadratique dans
l'équation, ce qui donne
\[\fpart{v_{OUT}}{t} = - \frac{k_n}{C_{rout}} v_{OV} v_{OUT}.\]
La solution est donc du type
\[v_{OUT}(t) = \alpha\exp^{-t/\tau},\]
avec $\tau = C_{rout} / k_n / v_{OV}$ et $\alpha$ tel que
\[v_{OUT}(t_{SAT}) =  v_{OV}.\]
On trouve donc
\begin{align*}
    \tau &= TODO \\
    \alpha &= TODO \\
    t^* &= TODO \\
\end{align*}

\subparagraph{Sans simplification.} On peut aussi résoudre l'équation
par des méthodes numériques, ce qui donne TODO.

\subsection*{Question 3: Délai de propagation: simplification}

Si on remplace le transistor par $R_{eq}$, on a l'équation
\[v_{OUT}(t) = V_{DD} \exp^{-t/(R_{eq}C_{rout}},\]
donc \[\frac{t^*}{R_{eq}C_{rout}} = \ln 2.\]
On trouve $R_{eq} = TODO$.

Dans la zone linéaire (approximation linéaire du régime triode),
on avait $\tau = C_{rout} / (k_n v_{OV})$. Par conséquent,
$R_{lin} = 1 / (k_n v_{OV}) = TODO$.

Finalement,
\[k = \frac{R_{eq}}{R_{lin}} = TODO.\]

\subsection*{Question 4: Scaling de la technologie CMOS}

\[R_{eq} = k R_{lin} = \frac{k}{k_n (V_{DD} - V_t)} = TODO\]
\section*{Exercice 3}

\end{document}

