\documentclass[frenchb,DIV=14]{scrartcl}

\usepackage[utf8x]{inputenc}
\usepackage[T1]{fontenc}
\usepackage{lmodern}
\usepackage{microtype}
\usepackage{xspace}
\usepackage[binary-units=true]{siunitx}
%\usepackage{color}
%\usepackage[usenames,dvipsnames,svgnames,table]{xcolor}
\usepackage{graphicx}
\usepackage{hyperref}
\usepackage{todonotes}
\usepackage{epstopdf}
\usepackage{array}
\usepackage{multicol}
\usepackage{multirow}
\usepackage{tabularx} % tabular with automatic line-break
\newcolumntype{Y}{>{\centering\arraybackslash}X} % centered column
\usepackage{ifthen}
\usepackage{amsmath}
\usepackage{grffile} % better name handling with graphicx
\usepackage{currfile} % provides relative file inclusion for tikzscale

\usepackage{tikz}
\usepackage{pgfplots}
\usepackage{tikz-3dplot}
\usepackage{pgfplots}
\usepackage{tikzscale}
\pgfplotsset{compat=newest}
\usetikzlibrary{plotmarks}
\usepackage{rotating}

\usepackage[french]{babel}
% Listing
% always put it after babel
% http://tex.stackexchange.com/questions/100717/code-in-lstlisting-breaks-document-compile-error
\usepackage{listings}

\definecolor{mygreen}{rgb}{0,0.6,0}
\definecolor{mygray}{rgb}{0.5,0.5,0.5}
\definecolor{mymauve}{rgb}{0.58,0,0.82}
\lstset{ %
  language=Matlab,
  backgroundcolor=\color{white},   % choose the background color; you must add \usepackage{color} or \usepackage{xcolor}
  basicstyle=\footnotesize,        % the size of the fonts that are used for the code
  breakatwhitespace=false,         % sets if automatic breaks should only happen at whitespace
  breaklines=true,                 % sets automatic line breaking
  captionpos=b,                    % sets the caption-position to bottom
  commentstyle=\color{mygreen},    % comment style
  deletekeywords={...},            % if you want to delete keywords from the given language
  escapeinside={\%*}{*)},          % if you want to add LaTeX within your code
  extendedchars=true,              % lets you use non-ASCII characters; for 8-bits encodings only, does not work with UTF-8
  frame=single,	                   % adds a frame around the code
  keepspaces=true,                 % keeps spaces in text, useful for keeping indentation of code (possibly needs columns=flexible)
  keywordstyle=\color{blue},       % keyword style
  otherkeywords={*,...},           % if you want to add more keywords to the set
  numbers=none,                    % where to put the line-numbers; possible values are (none, left, right)
  numbersep=5pt,                   % how far the line-numbers are from the code
  numberstyle=\tiny\color{mygray}, % the style that is used for the line-numbers
  rulecolor=\color{black},         % if not set, the frame-color may be changed on line-breaks within not-black text (e.g. comments (green here))
  showspaces=false,                % show spaces everywhere adding particular underscores; it overrides 'showstringspaces'
  showstringspaces=false,          % underline spaces within strings only
  showtabs=false,                  % show tabs within strings adding particular underscores
  stepnumber=2,                    % the step between two line-numbers. If it's 1, each line will be numbered
  stringstyle=\color{mymauve},     % string literal style
  tabsize=2,	                   % sets default tabsize to 2 spaces
  title=\lstname                   % show the filename of files included with \lstinputlisting; also try caption instead of title
}

\newcommand{\matlab}{\textsc{Matlab}}

% Math symbols
\usepackage{amsmath}
\usepackage{amssymb}
\usepackage{amsthm}
\DeclareMathOperator*{\argmin}{arg\,min}
\DeclareMathOperator*{\argmax}{arg\,max}


% Sets
\newcommand{\Z}{\mathbb{Z}}
\newcommand{\R}{\mathbb{R}}
\newcommand{\Rn}{\R^n}
\newcommand{\Rnn}{\R^{n \times n}}
\newcommand{\C}{\mathbb{C}}
\newcommand{\K}{\mathbb{K}}
\newcommand{\Kn}{\K^n}
\newcommand{\Knn}{\K^{n \times n}}

% Unit vectors
\usepackage{esint}
\usepackage{esvect}
\newcommand{\kmath}{k}
\newcommand{\xunit}{\hat{\imath}}
\newcommand{\yunit}{\hat{\jmath}}
\newcommand{\zunit}{\hat{\kmath}}
\newcommand{\uunit}{\hat{\umath}}

% rot & div & grad & lap
\DeclareMathOperator{\newdiv}{div}
\newcommand{\divn}[1]{\nabla \cdot #1}
\newcommand{\rotn}[1]{\nabla \times #1}
\newcommand{\grad}[1]{\nabla #1}
\newcommand{\gradn}[1]{\nabla #1}
\newcommand{\lap}[1]{\nabla^2 #1}


% Elec
\newcommand{\B}{\vec B}
\newcommand{\E}{\vec E}
\newcommand{\EMF}{\mathcal{E}}
\newcommand{\perm}{\varepsilon} % permittivity

\newcommand{\bigoh}{\mathcal{O}}
\newcommand\eqdef{\triangleq}

\DeclareMathOperator{\newdiff}{d} % use \dif instead
\newcommand{\dif}{\newdiff\!}
\newcommand{\fpart}[2]{\frac{\partial #1}{\partial #2}}
\newcommand{\ffpart}[2]{\frac{\partial^2 #1}{\partial #2^2}}
\newcommand{\fdpart}[3]{\frac{\partial^2 #1}{\partial #2\partial #3}}
\newcommand{\fdif}[2]{\frac{\dif #1}{\dif #2}}
\newcommand{\ffdif}[2]{\frac{\dif^2 #1}{\dif #2^2}}
\newcommand{\constant}{\ensuremath{\mathrm{cst}}}

%\KOMAoptions{DIV=last}

\usepackage{circuitikz}
\usetikzlibrary{babel}
\ctikzset{tripoles/mos style/arrows}

\makeatletter
\pgfcircdeclaremos{pmos}{
          \anchor{S}{
            \northeast
          }
          \anchor{source}{
            \northeast
          }
          \anchor{D}{
            \northeast
            \pgf@y=-\pgf@y
          }
          \anchor{drain}{
            \northeast
            \pgf@y=-\pgf@y
          }
}{%
            \pgfpathmoveto{\pgfpoint{\pgf@circ@res@right}{\pgf@circ@res@up}}
            \pgfpathlineto{\pgfpoint{\pgf@circ@res@right}
                {\pgfkeysvalueof{/tikz/circuitikz/tripoles/pmos/gate height}\pgf@circ@res@up}}
            \pgfpathlineto{\pgfpoint
                {\pgfkeysvalueof{/tikz/circuitikz/tripoles/pmos/base width}\pgf@circ@res@left}
                {\pgfkeysvalueof{/tikz/circuitikz/tripoles/pmos/gate height}\pgf@circ@res@up}}
            \pgfusepath{draw}

        \ifpgf@circuit@mos@arrows
            \pgfscope             
            \pgfslopedattimetrue 
            \pgfallowupsidedownattimetrue
            \pgfresetnontranslationattimefalse
            \pgftransformlineattime{.4}{%
                \pgfpoint%
                    {\pgf@circ@res@right}%
                    {\pgfkeysvalueof{/tikz/circuitikz/tripoles/pmos/gate height}\pgf@circ@res@up}%
            }{%
                \pgfpoint
                    {\pgfkeysvalueof{/tikz/circuitikz/tripoles/pmos/gate width}\pgf@circ@res@left}%
                    {\pgfkeysvalueof{/tikz/circuitikz/tripoles/pmos/gate height}\pgf@circ@res@up}%
            }
            \pgfnode{currarrow}{center}{}{}{\pgfusepath{stroke}}
            \endpgfscope
        \fi

            \pgfscope
            \pgfpathmoveto{\pgfpoint
                {\pgfkeysvalueof{/tikz/circuitikz/tripoles/pmos/base width}\pgf@circ@res@left}
                {\pgfkeysvalueof{/tikz/circuitikz/tripoles/pmos/base height}\pgf@circ@res@up}}
            \pgfpathlineto{\pgfpoint
                {\pgfkeysvalueof{/tikz/circuitikz/tripoles/pmos/base width}\pgf@circ@res@left}
                {\pgfkeysvalueof{/tikz/circuitikz/tripoles/pmos/base height}\pgf@circ@res@down}}
            \pgfsetlinewidth{2\pgflinewidth}
            \pgfusepath{draw}
            \endpgfscope

            \pgfpathmoveto{\pgfpoint
                {\pgfkeysvalueof{/tikz/circuitikz/tripoles/pmos/base width}\pgf@circ@res@left}
                {\pgfkeysvalueof{/tikz/circuitikz/tripoles/pmos/gate height}\pgf@circ@res@down}}
            \pgfpathlineto{\pgfpoint{\pgf@circ@res@right}
                {\pgfkeysvalueof{/tikz/circuitikz/tripoles/pmos/gate height}\pgf@circ@res@down}}
            \pgfpathlineto{\pgfpoint{\pgf@circ@res@right}{\pgf@circ@res@down}}      

            \pgfpathmoveto{\pgfpoint
                {\pgfkeysvalueof{/tikz/circuitikz/tripoles/pmos/gate width}\pgf@circ@res@left}
                {\pgfkeysvalueof{/tikz/circuitikz/tripoles/pmos/gate height}\pgf@circ@res@up}}
            \pgfpathlineto{\pgfpoint
                {\pgfkeysvalueof{/tikz/circuitikz/tripoles/pmos/gate width}\pgf@circ@res@left}
                {\pgfkeysvalueof{/tikz/circuitikz/tripoles/pmos/gate height}\pgf@circ@res@down}}


            \pgfpathmoveto{\pgfpoint
                {\pgfkeysvalueof{/tikz/circuitikz/tripoles/pmos/gate width}\pgf@circ@res@left}
                {\pgf@circ@res@up+\pgf@circ@res@down}}
            \pgfpathlineto{\pgfpoint{\pgf@circ@res@left}{\pgf@circ@res@up+\pgf@circ@res@down}}
            \pgfusepath{draw}

%           \pgfpathcircle{\pgfpoint
%               {\pgfkeysvalueof{/tikz/circuitikz/tripoles/pmos/gate width}\pgf@circ@res@left - \pgfkeysvalueof{/tikz/circuitikz/nodes width}*\pgfkeysvalueof{/tikz/circuitikz/bipoles/length}}
%               {\pgf@circ@res@up+\pgf@circ@res@down}}{\pgfkeysvalueof{/tikz/circuitikz/nodes width}*\pgfkeysvalueof{/tikz/circuitikz/bipoles/length}}
%           \pgfusepath{draw,fill}      

}

\titlehead{}
\subject{LELEC1530}
\title{Séance 5 - MOSFET: analyse fréquentielle}
\subtitle{Solutions}
\author{\small Gaëtan \textsc{Cassiers} \and\small Antoine \textsc{Paris}}
\date{}

\begin{document}
\maketitle

\section*{Exercice 1}

Le modèle petit-signal (en négligeant $r_0$) est donné à la figure~\ref{fig:pt-signal1}.

% Note, ce serait mieux d'utiliser ls schéma en T comme dans le cas général la source
% n'est pas à la masse en petit signal.
\begin{figure}[h]
    \centering
    \subfloat[Circuit complet]{%
        \begin{tikzpicture}
            \draw (0, 0) to [R=$R_S$] (0, 2)
            (0, 4) to[cI=$g_m v_{gs}$] (0, 2)
            (0, 2) -- (1, 2) to[C=$C_S$] (1, 0) -- (0, 0)
            (0, 4) -- (2.5, 4) to[R=$R_D$] (2.5, 0) -- (1, 0)
            (2.5, 4) to[short, -o] (3, 4) node[anchor=west] {$v_{out}$}
            (-1, 4) to [short, -o] (-1.5, 4) to[open, v=$v_{gs}$] (-1.5, 2) to[short, o-] (0, 2)
            (0, 0) node[ground] {};
        \end{tikzpicture}
        \label{fig:pt-signal1-full}}
    \subfloat[Circuit dans la bande passante]{%
        \begin{circuitikz}
            \draw (0, 0) to[open] (0, 2)
            (0, 4) to[cI=$g_m v_{gs}$] (0, 2)
            (0, 2) -- (1, 2) to[short] (1, 0) -- (0, 0)
            (0, 4) -- (2.5, 4) to[R=$R_D$] (2.5, 0) -- (1, 0)
            (2.5, 4) to[short, -o] (3, 4) node[anchor=west] {$v_{out}$}
            (-1, 4) to [short, -o] (-1.5, 4) to[open, v=$v_{gs}$] (-1.5, 2) to[short, o-] (0, 2)
            (0, 0) node[ground] {};
        \end{circuitikz}
        \label{fig:pt-signal1-bp}}
    \caption{Modèle petit-signal du circuit}
    \label{fig:pt-signal1}
\end{figure}

\subsection*{Gain en bande passante}

Dans la bande passante, la fréquence est suffisamment grande pour que
l'admittance $1/R_S$ soit négligeable par rapport à l'admittance
$j\omega C_S$. On peut donc simplifier le schéma petit signal pour trouver
le circuit de la figure~\ref{fig:pt-signal1-bp}.

On touve alors:
\[v_{out} = -R_D i_d = -R_D g_m v_{gs} = -R_D g_m v_{in}\]
et donc le gain vaut
\[A_v = \frac{v_{out}}{v_{in}} = -R_D g_m = -10.\]

\paragraph{Remarque}
$C_s$ est ce qu'on appelle un \emph{bypass capacitor}. Il permet ici d'amener
la source à la masse en petit signal, ce qui augmente le gain $A_v$. En effet,
sans $C_s$, on obtient
\[A'_v = -\frac{-R_D g_m}{1+g_m R_s} = -1.43.\]

\subsection*{Fonction de transfert}

On se base sur la figure~\ref{fig:pt-signal1-full} pour trouver les équations du circuit:
\begin{align*}
    v_{out} &= -R_D i_d \\
    i_d &= g_m v_{gs} \\
    v_{gs} &= v_{in} - \frac{1}{\frac{1}{R_S} + sC_S} i_d \\
\end{align*}

À partir de ces équations, on trouve
\begin{align*}
    v_s &= \frac{g_m}{g_m + \frac{1}{R_S} + s C_S} v_{in} \\
    v_{out} &= -R_D g_m \left(v_{in} - v_s\right) \\
\end{align*}
et
% TODO: voir si c'est mieux de noter A_v ou A'_v dans cette équation
\begin{align*}
    H(s) = \frac{v_{out}}{v_{in}}
        &= \overbrace{-g_m R_D}^{A_v}\frac{1 + s C_S R_S}{1 + g_m R_S + s C_S R_S} \\
        &= -10\frac{1 + s C_S R_S}{7 + s C_S R_S}. \\
\end{align*}

\subsection*{Diagramme de Bode}

Ci-dessous le diagramme de Bode (en prenant $C_S = \SI{100}{\micro\farad}$).
\begin{center}
    \includegraphics[width=0.7\textwidth,height=0.3\textwidth]{figures/bode1.tikz}
    \includegraphics[width=0.7\textwidth,height=0.3\textwidth]{figures/bode2.tikz}
\end{center}

% TODO: très petit rappel sur comment plotter ça à la main? (remise sous la bonne forme,
% passage en log, et brève explication?).

\subsection*{Gain DC}
En DC, $\omega = 0$:
\[H(0) = -\frac{10}{7} = -1.43\]

\section*{Exercice 2}

\subsection*{Point de polarisation}

Sachant que (en supposant que le transistor est en saturation)
\begin{align*}
    I_D &= k_n \frac{W}{L} \frac{\left(V_G - V_S - V_{T0}\right)^2}{2} \\
    V_G &= \frac{V_{DD}}{2} \\
    V_S &= R_4 I_D, \\
\end{align*}
on trouve \[\frac{W}{L} = 2.\]

On a aussi $V_{OUT} = V_{DD} - R_3 I_D = \SI{2.5}{V}$, donc $R_3 = \SI{2.5}{k\ohm}$.

Comme $V_{OV} = \SI{0.5}{V} > 0$ et $V_{DS} = \SI{1.5}{V} > V_{OV}$, le transistor
est saturé; l'équation utilisée ci-dessus est donc correct.

\subsection*{Gain et impédances d'entrée et de sortie en bande passante}
\begin{figure}[h]
    \centering
        \begin{tikzpicture}
            \draw
            (0, 4) to[cI_=$g_m v_{gs}$] (0, 2)
            (0, 0) to[V=$v_{in}$] (0, 2)
            (0, 0) node[ground] {}
            (0, 4) -- (1, 4) to[R=$r_0$] (1, 2) to [short, -*] (0, 2)
            (1, 2) to[R=$R_4$] (1, 0) -- (0, 0)
            (1, 4) -- (2.5, 4) to[R=$R_3$] (2.5, 0) -- (1, 0)
            (2.5, 4) to[short, -o] (3, 4) node[anchor=west] {$v_{out}$}
            ;
        \end{tikzpicture}
    \caption{Modèle petit-signal du circuit dans la bande passante ($v_{gs} = -v_{in}$)}
    \label{fig:pt-signal2}
\end{figure}
Sur le modèle petit-signal,
\begin{align*}
    g_m &= \frac{2I_D}{V_{OV}} = \SI{4}{mA/V} \\
    r_0 &= \frac{1}{\lambda I_D} = \SI{50}{k\ohm}.
\end{align*}

En se basant sur le schéma petit-signal (figure~\ref{fig:pt-signal2}), on
trouve le gain (par exemple via un équivalent de Thévenin de la source de courant
et de $r_0$, ou via KCL au noeud de sortie):
\[A_v = \frac{v_{out}}{v_{in}} = (1+g_m r_0) \frac{R_3}{r_0 + R_3}\]

La résistance d'entrée s'obtient en laissant la sortie en circuit ouvert.
On trouve alors
\[i_{in} = \left(\frac{1}{R_4} + \frac{1+r_0 g_m}{r_0 + R_3}\right) v_{in}.\]
D'où
\[R_{in} = \frac{R_4 (r_0 + R_3)}{r_0 + R_3 + R_4 (1+r_0 g_m)}.\]

La résistance de sortie s'obtient en imposant $v_{in} = 0$:
\[R_{out} = R_3 \parallelsum r_0.\]

\subsection*{Bande passante}
Le modèle petit signal complet est donné à la figure~\ref{fig:pt-signal3}.

\begin{figure}[h]
    \centering
        \begin{tikzpicture}
            \draw
            (0, 4) to[cI_=$g_m v_{gs}$] (0, 2)
            (0, 0) to[V=$v_{in}$] (0, 2)
            (0, 0) node[ground] {}
            (0, 4) -- (1, 4) to[R=$r_0$] (1, 2) to [short, -*] (0, 2)
            (1, 2) to[R=$R_4$] (1, 0) -- (0, 0)
            (1, 4) -- (2.5, 4) to[R=$R_3$] (2.5, 0) -- (1, 0)
            (2.5, 4) to[short, -o] (4.5, 4) node[anchor=west] {$v_{out}$}
            (0, 2) -- (-2,2) to[C] node[anchor=east] {$C_{GS}$} (-2,0) to [short] (0,0)
            (4, 4) to[C=$C_{GD}$] (4,0) to [short] (2.5, 0)
            ;
        \end{tikzpicture}
    \caption{Modèle petit-signal complet du circuit}
    \label{fig:pt-signal3}
\end{figure}

On peut réecrire ce circuit de manière à mieux faire apparaître l'effet Miller
(figure~\ref{fig:pt-signal4}).

\begin{figure}[h]
	\centering
		\begin{tikzpicture}
			\draw
			(2,0) -- (0,0) to[V=$v_{in}$] (0,2) -- (2,2)
			(4,0) -- (2,0) to[R=$R_4$] (2,2) -- (4,2)
			(2,0) node[ground] {}
			(4,0) to[C=$C_{GS}$] (4,2)
			(4,2) -- (5,2)
			(5,2) -- (5,3) to[R=$r_0$] (7,3) -- (7,2)
			(7,2) -- (7,1) to[cI=$g_m v_{gs}$] (5,1) -- (5,2)
			(7,2) -- (8,2) to[R=$R_3$] (8,0)
			(8,2) -- (10,2) to[C=$C_{GS}$] (10,0) -- (8,0) 
			(9,0) node[ground] {}
			(10,2) to[short, -o] (10.5,2) node[anchor=west] {$v_{out}$}
			;
		\end{tikzpicture}
	\caption{Modèle petit signal complet du circuit réécrit. La résistance $r_0$
	entre l'entrée et la sortie cause l'effet Miller.}
	\label{fig:pt-signal4}
\end{figure}

On applique ensuite le théorème de Miller sur le circuit de la figure~\ref{fig:pt-signal4}
pour obtenir le circuit de la figure~\ref{fig:pt-signal5}.
Si $K$ est le gain entre $v_{in}$ et $v_{out}$, le théorème de Miller nous dit
\begin{align*}
	r_1 &= r_0(1-K) & r_2 &= r_0(1-\frac{1}{K}).
\end{align*}

\begin{figure}[h]
	\centering
		\begin{tikzpicture}
			\draw
			(2,0) -- (0,0) to[V=$v_{in}$] (0,2) -- (2,2)
			(4,0) -- (2,0) to[R=$R_4$] (2,2) -- (4,2)
			(6,0) -- (4,0) to[C=$C_{GS}$] (4,2) -- (6,2)
			(6,0) to[R=$r_1$] (6,2)
			(3,0) node[ground] {}
			(8,2) to[cI=$g_m v_{gs}$] (6,2)
			(8,2) to[R=$r_2$] (8,0)
			(8,0) -- (10,0) to[R=$R_3$] (10,2) -- (8,2)
			(10,2) -- (12,2) to[C=$C_{GS}$] (12,0) -- (10,0) 
			(10,0) node[ground] {}
			(12,2) to[short, -o] (12.5,2) node[anchor=west] {$v_{out}$}
			;
		\end{tikzpicture}
	\caption{Modèle petit signal complet du circuit équivalent en utilisant le
	théorème de Miller.}
	\label{fig:pt-signal5}
\end{figure}

Enfin, on peut encore transformer le circuit de la figure~\ref{fig:pt-signal5}
en se rendant compte que l'effet de la source de courant commandée sur le circuit
se retrouve bien également sur le circuit de la figure~\ref{fig:pt-signal6}.

\begin{figure}[h]
	\centering
		\begin{tikzpicture}[scale=0.92]
			\draw
			(2,0) -- (0,0) to[V=$v_{in}$] (0,2) -- (2,2)
			(4,0) -- (2,0) to[R=$R_4$] (2,2) -- (4,2)
			(6,0) -- (4,0) to[C=$C_{GS}$] (4,2) -- (6,2)
			(8,0) -- (6,0) to[R=$r_1$] (6,2) -- (8,2)
			(8,0) to[R=$1/g_m$] (8,2)
			(4,0) node[ground] {}
			
			(12,0) -- (10,0) to[cI=$g_m v_{in}$] (10,2) -- (12,2)
			(14,0) -- (12,0) to[R=$r_2$] (12,2) -- (14,2)
			(16,0) -- (14,0) to[R=$R_3$] (14,2) -- (16,2)
			(16,0) to[C=$C_{GS}$] (16,2) 
			(14,0) node[ground] {}
			(16,2) to[short, -o] (16.5,2) node[anchor=west] {$v_{out}$}
			;
		\end{tikzpicture}
	\caption{Modèle petit signal complet du circuit équivalent final.}
	\label{fig:pt-signal6}
\end{figure}

La fonction de transfert entre $v_{in}$ et $v_{out}$ est maintenant beaucoup
plus facileà obtenir qu'initialement
\[ v_{out} = g_m v_{in} (r_2 \parallelsum R_3 \parallelsum C_{GD}) \]
où
\[ r_2 = r_0\left(1-\frac{g_3+g_d}{g_m+g_d}\right) \]
avec $g_3 = 1/R_3$, $g_d = 1/r_0$.
La fonction de transfert est alors donnée par
\[ H(s) = \frac{v_{out}}{v_{in}} = \frac{g_m}{g_2 + g_3 + sC_{GD}}. \]

\section*{Exercice 3}
\subsection*{Gain en bande passante}
Le circuit petit signal est donné à la figure~\ref{fig:pt-signal7}.
\begin{figure}[h]
	\centering
		\begin{tikzpicture}[scale=0.92]
			\draw
			(0, 2) to[cI=$g_m v_{in}$] (0,0)
			(0,2) -- (2.5,2) to[R=$r_0$] (2.5,0) -- (0,0)
			(1.5,0) node[ground]{}
			(2.5,2) -- (4.5,2) node[anchor=south] {$v_{s2}$}
			(4.5,2) to[cI=$g_m v_{s2}$] (4.5,0)
			(4.5,2) -- (7,2) to[R=$r_0$] (7,0) -- (4.5,0)
			(7,0) to[R=$r_0$] (7,-2)
			(7,-2) node[ground]{}
			(7,0) to[short, -o] (8, 0) node[anchor=west] {$v_{out}$}
			;
		\end{tikzpicture}
	\caption{Modèle petit signal de l'amplificateur cascode replié}
	\label{fig:pt-signal7}
\end{figure}

On écrit ensuite KCL au noeud de sortie
\[ g_m v_{s2} = \frac{v_{out}-v_{s2}}{r_0} + \frac{v_{out}}{r_0}.\]
Avec $v_{s2} = -g_m v_{in} r_0$, on trouve finalement
\[ \frac{v_{out}}{v_{in}} = -0.5g_m r_0(1+g_mr_0). \]
% TODO: différence par rapport à l'énoncé?

\subsection*{Pôle haute fréquence}
On tient cette fois compte de la capacité de charge $C_L$ en sortie du montage,
le schéma petit signal complet est donné la figure~\ref{fig:pt-signal8}.

\begin{figure}[h]
	\centering
		\begin{tikzpicture}[scale=0.92]
			\draw
			(0, 2) to[cI=$g_m v_{in}$] (0,0)
			(0,2) -- (2.5,2) to[R=$r_0$] (2.5,0) -- (0,0)
			(1.5,0) node[ground]{}
			(2.5,2) -- (4.5,2) node[anchor=south] {$v_{s2}$}
			(4.5,2) to[cI=$g_m v_{s2}$] (4.5,0)
			(4.5,2) -- (7,2) to[R=$r_0$] (7,0) -- (4.5,0)
			(7,0) to[R=$r_0$] (7,-2)
			(8,-2) node[ground]{}
			(7,0) to[short, -o] (10, 0) node[anchor=west] {$v_{out}$}
			(9,0) to[C=$C_L$] (9,-2) -- (7,-2)
			;
		\end{tikzpicture}
	\caption{Modèle petit signal de l'amplificateur cascode replié avec la capacité
	de charge.}
	\label{fig:pt-signal8}
\end{figure}

On réecrit KCL au noeud de sortie
\[ g_m v_{s2} = \frac{v_{out}-v_{s2}}{r_0} + \frac{v_{out}}{r_0} + v_{out}sC_L. \]
En utilisant $v_{s2} = -g_m v_{in} r_0$, on obtient finalement
\[ H(s) = \frac{v_{out}}{v_{in}} = -\frac{g_mr_0}{2}\frac{1+g_mr_0}{1+s\frac{r_0C_L}{2}}\]
et donc
\[ \omega_H = \frac{2}{r_0C_L} \]
est bien un pôle haute fréquence.

\end{document}