\documentclass[frenchb,DIV=13]{scrartcl}

\usepackage[utf8x]{inputenc}
\usepackage[T1]{fontenc}
\usepackage{lmodern}
\usepackage[binary-units=true]{siunitx}
% Color
% cfr http://en.wikibooks.org/wiki/LaTeX/Colors
\usepackage{color}
\usepackage[usenames,dvipsnames,svgnames,table]{xcolor}
\definecolor{dkgreen}{rgb}{0.25,0.7,0.35}
\definecolor{dkred}{rgb}{0.7,0,0}
\usepackage{graphicx}
\usepackage{url}
\usepackage{tikz}
\usepackage{pgfplots}
\usepackage{microtype}
\usepackage{xspace}

\usepackage{hyperref}
\usepackage{todonotes}
\usepackage{epstopdf}

\usepackage{multirow}
\usepackage{tabularx} % tabular with automatic line-break
\newcolumntype{Y}{>{\centering\arraybackslash}X} % centered column

\newcommand{\matlab}{\textsc{Matlab}}

% Math symbols
\usepackage{amsmath}
\usepackage{amssymb}
\usepackage{amsthm}
\DeclareMathOperator*{\argmin}{arg\,min}
\DeclareMathOperator*{\argmax}{arg\,max}

% Unit vectors
\usepackage{esint}
\usepackage{esvect}
\newcommand{\kmath}{k}
\newcommand{\xunit}{\hat{\imath}}
\newcommand{\yunit}{\hat{\jmath}}
\newcommand{\zunit}{\hat{\kmath}}
\newcommand{\uunit}{\hat{\umath}}

% Elec
\newcommand{\B}{\vec B}
\newcommand{\E}{\vec E}
\newcommand{\EMF}{\mathcal{E}}
\newcommand{\perm}{\varepsilon} % permittivity

\newcommand{\bigoh}{\mathcal{O}}
\newcommand\eqdef{\triangleq}

\DeclareMathOperator{\newdiff}{d} % use \dif instead
\newcommand{\dif}{\newdiff\!}
\newcommand{\fpart}[2]{\frac{\partial #1}{\partial #2}}
\newcommand{\ffpart}[2]{\frac{\partial^2 #1}{\partial #2^2}}
\newcommand{\fdpart}[3]{\frac{\partial^2 #1}{\partial #2\partial #3}}
\newcommand{\fdif}[2]{\frac{\dif #1}{\dif #2}}
\newcommand{\ffdif}[2]{\frac{\dif^2 #1}{\dif #2^2}}
\newcommand{\constant}{\ensuremath{\mathrm{cst}}}
\newcommand{\norm}[1]{\left\lVert#1\right\rVert}

\usepackage{babel}
% Listing
% always put it after babel
% http://tex.stackexchange.com/questions/100717/code-in-lstlisting-breaks-document-compile-error
\usepackage{listings}

% Put caption after babel
\usepackage{caption}
\usepackage{subcaption}

\definecolor{mygreen}{rgb}{0,0.6,0}
\definecolor{mygray}{rgb}{0.5,0.5,0.5}
\definecolor{mymauve}{rgb}{0.58,0,0.82}
\lstset{ %
  language=Matlab,
  backgroundcolor=\color{white},   % choose the background color; you must add \usepackage{color} or \usepackage{xcolor}
  basicstyle=\footnotesize,        % the size of the fonts that are used for the code
  breakatwhitespace=false,         % sets if automatic breaks should only happen at whitespace
  breaklines=true,                 % sets automatic line breaking
  captionpos=b,                    % sets the caption-position to bottom
  commentstyle=\color{mygreen},    % comment style
  deletekeywords={...},            % if you want to delete keywords from the given language
  escapeinside={\%*}{*)},          % if you want to add LaTeX within your code
  extendedchars=true,              % lets you use non-ASCII characters; for 8-bits encodings only, does not work with UTF-8
  frame=single,	                   % adds a frame around the code
  keepspaces=true,                 % keeps spaces in text, useful for keeping indentation of code (possibly needs columns=flexible)
  keywordstyle=\color{blue},       % keyword style
  otherkeywords={*,...},           % if you want to add more keywords to the set
  numbers=left,                    % where to put the line-numbers; possible values are (none, left, right)
  numbersep=5pt,                   % how far the line-numbers are from the code
  numberstyle=\tiny\color{mygray}, % the style that is used for the line-numbers
  rulecolor=\color{black},         % if not set, the frame-color may be changed on line-breaks within not-black text (e.g. comments (green here))
  showspaces=false,                % show spaces everywhere adding particular underscores; it overrides 'showstringspaces'
  showstringspaces=false,          % underline spaces within strings only
  showtabs=false,                  % show tabs within strings adding particular underscores
  stepnumber=1,                    % the step between two line-numbers. If it's 1, each line will be numbered
  stringstyle=\color{mymauve},     % string literal style
  tabsize=2,	                   % sets default tabsize to 2 spaces
  title=\lstname                   % show the filename of files included with \lstinputlisting; also try caption instead of title
}

\KOMAoptions{DIV=last}




\titlehead{}
\subject{LELEC1530}
\title{Séance 3 - BJT: polarisation et amplificateur}
\subtitle{Solutions}
\author{\small Gaëtan \textsc{Cassiers} \and\small Antoine \textsc{Paris}}
\date{}

\begin{document}
\maketitle

\section*{Exercice 1}
Chacun des circuits proposés, $|V_{BE}| = \SI{0.7}{\volt}$. Le transistor peut
donc être en régime actif ou en saturation selon la valeur de $V_{BC}$ ou de
$V_{CB}$ pour un NPN ou PNP respectivement.

Dans tous les cas, considérer le gain $\beta$ comme très grand revient à considérer
$I_B$ comme nul et donc à considérer $I_C = I_E$.

\subsection*{Circuit a)} 
Il s'agit d'un transistor bipolaire de type PNP. La flèche indique l'émetteur et
on voit que $V_E = \SI{0.7}{\volt}$ tandis que $V_B = \SI{0}{\volt}$.
Le dernier terminal est le collecteur ($V_C$).

On a
\[ I_E = I_C = \frac{10.7-0.7}{\SI{10e3}{}} = \SI{1}{\milli\ampere} \]
et donc
\[ V_C = \SI{-0.7}{\volt}. \]
La tension $V_{CB} = \SI{-0.7}{\volt}$ est $< \SI{0.4}{\volt}$ et le transistor
est donc en mode actif.

\subsection*{Circuit b)}
Il s'agit d'un transistor bipolaire de type NPN. La flèche indique l'émetteur.
Comme $I_B$ est nul, il n'y a pas de chute de tension sur la résistance reliant
la base au collecteur et donc $V_B = V_C \Rightarrow V_{BC} = \SI{0}{\volt} < \SI{0.4}{\volt}$:
le transistor est donc en mode actif. On a ensuite
\begin{align*}
	I_C &= I_E \\
	\frac{10-V_C}{\SI{10e3}{}} &= \frac{V_E - (-10)}{\SI{10e3}{}} \\
	10 - (V_E + 0.7) &= V_E + 10 \\
	V_E &= -\SI{0.35}{\volt}
\end{align*}
et donc $V_C = V_B = V_E+0.7 = \SI{0.35}{\volt}$ et $I_E = I_C = \SI{965}{\micro\ampere}$.

\subsection*{Circuit c)}
Il s'agit d'un transistor bipolaire de type PNP. La flèche indique l'émetteur. On commence
par calculer la tension à la bas $V_B$
\begin{align*}
	\frac{10-V_B}{\SI{180e3}{}} &= \frac{V_B-(-10)}{\SI{300e3}{}} \\
	V_B & = \SI{2.5}{\volt}.
\end{align*}
Le courant dans le branche de gauche vaut donc $\frac{10-2.5}{\SI{300}{e3}} = \SI{41.7}{\micro\ampere}$.
Comme $V_{EB} = \SI{0.7}{\volt}$, $V_E = \SI{3.2}{\volt}$.
On a ensuite
\[ I_E = I_C = \frac{10-3.2}{\SI{6.8e3}{}} = \SI{1}{\milli\ampere}.\]
Comme $I_C = \frac{V_C-10}{\SI{10e3}{}}$, on trouve finalement $V_C = \SI{0}{\volt}$.
$V_CB = -\SI{2.5}{\volt < \SI{0.4}{\volt}}$, le transistor est en mode actif.

\section*{Exercice 2}
\subsection*{$\beta = \infty$}
Comme pour l'exercice précédent, $\beta = \infty \Rightarrow I_B = 0, I_C = I_E$.
On a donc directement $I_{B1} = I_{B2} = \SI{0}{\ampere}$ et donc $V_{B1} = \SI{0}{\volt}$.

On procède ensuite de la même façon que pour les circuits impliquants des transistors MOS: on
fait des hypothèses sur le régime fonctionnement des transistors qu'on vérifie a posteriori.

Pour commencer, penchons-nous sur Q1 (PNP). Si Q1 est coupé (i.e. $V_{EB1} < \SI{0.7}{\volt}$
et $V_{CB1} < \SI{0.4}{\volt}$), aucun courant ne circule dans la branche de gauche. Dans ce cas,
il n'y a pas de chute de tension à travers les deux résistances de \SI{9.1}{\kilo\ohm} et donc
$V_{E1} = \SI{10}{\volt}$ et $V_{C1} = -\SI{10}{\volt}$. On a donc $V_{EB1} = \SI{10}{\volt}$
ce qui est en contradiction avec l'hyptohèse du mode coupé. Q1 est donc soit en mode actif
(i.e. $V_{EB1} \approx \SI{0.7}{\volt}$ et $V_{CB1} < \SI{0.4}{\volt}$) soit en mode saturé
(i.e. $V_{EB1} \approx \SI{0.7}{\volt}$ et $V_{CB1} \approx \SI{0.5}{\volt}$). Dans tous les cas,
on a $V_{EB1} \approx \SI{0.7}{\volt} \Rightarrow V_{E1} = \SI{0.7}{\volt}$. Ensuite,
\[ I_{E1} = I_{C1} = \frac{10-0.7}{\SI{9.1e3}{}} = \SI{1.02}{\milli\ampere}, \]
\[ V_{C1} = \SI{-0.7}{\volt} = V_{B2}. \]
A ce stade, on a assez d'information pour savoir dans quel mode se trouve le transistor. Comme
$V_{CB1} = \SI{-0.7}{\volt}$, Q1 est en mode actif. La branche de gauche est maintenant entièrement
connue, on passe à la branche de droite.

Q2 (NPN) peut-il être coupé (i.e. $V_{BE2} < \SI{0.7}{\volt}$ et $V_{BC2} < \SI{0.4}{\volt}$) ?
Si c'est le cas, on arrive à une contradiction exactement comme c'était le cas pour Q1.
Q1 est donc soit en mode actif (i.e. $V_{BE2} \approx \SI{0.7}{\volt}$ et $V_{BC2} < \SI{0.4}{\volt}$)
soit en mode saturé (i.e. $V_{BE2} \approx \SI{0.7}{\volt}$ et $V_{BC2} \approx \SI{0.5}{\volt}$). Dans
tous les cas, $V_{BE2} \approx \SI{0.7}{\volt}$. Or on sait que $V_{B2} = -\SI{0.7}{\volt}$, donc
$V_{E2} = -\SI{1.4}{\volt}$. Ensuite,
\[ I_{E2} = I_{C2} = \frac{-1.4-(-10)}{\SI{4.3e3}{}} = \SI{2}{\milli\ampere}, \]
\[ V_{C2} = -\SI{0.2}{\volt}. \]
Comme $V_{BC2} = -\SI{0.5}{\volt}$, Q2 et en mode actif. 

\subsection*{$\beta = 100$}
Cette fois, $\beta$ n'étant plus infini, on ne peut plus négliger $I_B$ et on a
\[ I_E = I_B + I_C. \]

On peut utiliser la même raisonnement que dans le cas précédent pour montrer que Q1
ne peut pas être coupé. On sait donc que $V_{EB1} \approx \SI{0.7}{\volt}$ mais,
contrairement au cas précédent, cette information seule ne permet pas de calculer
quoique ce soit de plus. Il faut donc faire une hypothèse de plus. On suppose ici que
Q1 est en mode actif (i.e. $V_{CB1} < \SI{0.4}{\volt}$). Dans ce cas, on a deux équations
de plus liant les courants qui nous permettent d'aller plus loin
\begin{align*}
	I_C &= \beta I_B & I_E &= (\beta+1)I_B.
\end{align*}
La dernière équation nous permet d'écrire
\[ \frac{10-V_{E1}}{\SI{9.1e3}{}} = (\beta+1)\frac{V_{B1}}{\SI{100e3}{}} \]
or, comme $V_{EB1} = V_{E1}-V_{B1} = \SI{0.7}{\volt}$, on peut réecrire cette
équation
\[ \frac{10-V_{E1}}{\SI{9.1e3}{}} = (\beta+1)\frac{V_{E1}-0.7}{\SI{100e3}{}}. \]
On trouve $V_{E1} = \SI{1.61}{\volt}$, et donc $V_{B1} = \SI{0.91}{\volt}$. Quand aux
courants, $I_{E1} = \SI{922}{\micro\ampere}$, $I_{B1} = \SI{9.12}{\micro\ampere}$ et
donc $I_{C1} = \SI{912}{\micro\ampere}$.

On suppose ensuite que Q2 est également en mode actif (i.e. $V_{BE2} \approx \SI{0.7}{\volt}$
et $V_{CE2} < \SI{0.4}{\volt}$. Ensuite, on utilise $I_E = (\beta+1)I_B$
\[ \frac{V_{E2}-(-10)}{\SI{4.3e3}{}} = (\beta+1)I_{B2}. \]
On a une seule équation et deux inconnues, il faut donc trouver unr autre équation. On
peut écrire KCL en $C1 = B2$
\[ I_{B2} = I_{C1} - \frac{V_{C1}-(-10)}{\SI{9.1e3}{}}. \]
Or comme $V_{C1} = V_{B2}$ et que, par hypothèse du mode actif, $V_{B2} = 0.7 + V_{E2}$,
on a finalement
\[ I_{B2} = I_{C1} - \frac{10.7 + V_{E2}}{\SI{9.1e3}{}}\]
que l'on peut injecter dans notre équation à 2 inconnues.
On touve finalement
\begin{align*}
	V_{E2} &= -\SI{2.55}{\volt} & V_{B2} &= V_{C1} = -\SI{1.85}{\volt}
\end{align*}
et
\begin{align*}
	I_{E2} &= \SI{1.73}{\milli\ampere} & I_{B2} &= -\SI{17.2}{\micro\ampere} & I_{C2} &=
	\SI{1.72}{\milli\ampere}
\end{align*}
et finalement
\[ V_{C2} = \SI{1.25}{\volt}. \]
On peut finalement vérifier les hypothèses de mode actif faites précédemment:
\begin{enumerate}
	\item $V_{CB1} = -\SI{2.76}{\volt} < \SI{0.4}{\volt}$ : OK.
	\item $V_{BC2} = -\SI{3.11}{\volt} < \SI{0.4}{\volt}$ : OK.
\end{enumerate}
 
\subsection*{$\beta = 10$}
Raisonnement identique au cas précédent.

\section*{Exercice 3}

\subsection*{Grand signal}

On calcule le courant \[I_C = \frac{V_C}{R_C} = \SI{4}{mA}.\]
Cette valeur sera utilisée pour le calcul de valeurs petit-signal.

On ne peut pas calculer les autres courants à ce stade car on ne connait pas $\beta$.

\subsection*{Courants en petit-signal}

Le circuit petit-signal est
\begin{center}
\begin{circuitikz}

    \draw (0, 2) to[R=$R_c$] (0, 4);
    \draw (0, 4) to[cI=$g_m\cdot v_{be}$] (0, 6) to[short] (1, 6) to[R=$r_0$] (1, 4) to[short] (0, 4);
    \draw (1, 6) to[short] (2, 6) to[short] (2, 2) to[short] (0, 2);
    \draw (0, 2) node[ground] {};

    \draw (-3, 6) to[R=$r_\pi$] (-3, 4) to[short] (-4, 4) to[V=$v_{eb}$] (-4, 6) to[short] (-3, 6) to[short] (0, 6);
    \draw (-0.5, 4) node[anchor=east] {$v_{out}$} to[short] (0, 4);

    \draw[very thick, dashed] [draw=black!60] (-3.4, 6.3) rectangle (1.8, 3.8);

\end{circuitikz}
\end{center}
avec $g_m = I_C/V_T = \SI{0.154}{A/V}$.

On trouve alors
\begin{align*}
    i_b(t) &= \frac{v_{eb}}{r_\pi} = v_{eb} \frac{g_m}{\beta} \\
    i_c(t) &= \beta i_b = v_{eb} g_m \\
    i_e(t) &= \frac{i_c}{\alpha} = i_c \frac{\beta+1}{\beta}
\end{align*}

\subsection*{Courants totaux}

\begin{align*}
    i_C(t) &= I_C + v_{eb} g_m \\
    i_B(t) &= \frac{i_C(t)}{\beta} \\
    i_E(t) &= \frac{i_C(t)}{\alpha}
\end{align*}

\subsection*{Gain}

D'après le schéma petit-signal, $v_{out} = (r_0 \parallelsum R_c) g_m v_{eb}$,
et donc $A_v = (r_0 \parallelsum R_c) g_m$.

Comme on ne connait pas $V_A$, on ne peut pas calculer $r_0$. On va donc négliger
l'effet Early.

On a donc $A_v = R_c g_m = 308$.

\end{document}
