\documentclass[frenchb,DIV=14]{scrartcl}

\usepackage[utf8x]{inputenc}
\usepackage[T1]{fontenc}
\usepackage{lmodern}
\usepackage{microtype}
\usepackage{xspace}
\usepackage[binary-units=true]{siunitx}
%\usepackage{color}
%\usepackage[usenames,dvipsnames,svgnames,table]{xcolor}
\usepackage{graphicx}
\usepackage{hyperref}
\usepackage{todonotes}
\usepackage{epstopdf}
\usepackage{array}
\usepackage{multicol}
\usepackage{multirow}
\usepackage{tabularx} % tabular with automatic line-break
\newcolumntype{Y}{>{\centering\arraybackslash}X} % centered column
\usepackage{ifthen}
\usepackage{amsmath}
\usepackage{grffile} % better name handling with graphicx
\usepackage{currfile} % provides relative file inclusion for tikzscale

\usepackage{tikz}
\usepackage{pgfplots}
\usepackage{tikz-3dplot}
\usepackage{pgfplots}
\usepackage{tikzscale}
\pgfplotsset{compat=newest}
\usetikzlibrary{plotmarks}
\usepackage{rotating}

\usepackage[french]{babel}
% Listing
% always put it after babel
% http://tex.stackexchange.com/questions/100717/code-in-lstlisting-breaks-document-compile-error
\usepackage{listings}

\definecolor{mygreen}{rgb}{0,0.6,0}
\definecolor{mygray}{rgb}{0.5,0.5,0.5}
\definecolor{mymauve}{rgb}{0.58,0,0.82}
\lstset{ %
  language=Matlab,
  backgroundcolor=\color{white},   % choose the background color; you must add \usepackage{color} or \usepackage{xcolor}
  basicstyle=\footnotesize,        % the size of the fonts that are used for the code
  breakatwhitespace=false,         % sets if automatic breaks should only happen at whitespace
  breaklines=true,                 % sets automatic line breaking
  captionpos=b,                    % sets the caption-position to bottom
  commentstyle=\color{mygreen},    % comment style
  deletekeywords={...},            % if you want to delete keywords from the given language
  escapeinside={\%*}{*)},          % if you want to add LaTeX within your code
  extendedchars=true,              % lets you use non-ASCII characters; for 8-bits encodings only, does not work with UTF-8
  frame=single,	                   % adds a frame around the code
  keepspaces=true,                 % keeps spaces in text, useful for keeping indentation of code (possibly needs columns=flexible)
  keywordstyle=\color{blue},       % keyword style
  otherkeywords={*,...},           % if you want to add more keywords to the set
  numbers=none,                    % where to put the line-numbers; possible values are (none, left, right)
  numbersep=5pt,                   % how far the line-numbers are from the code
  numberstyle=\tiny\color{mygray}, % the style that is used for the line-numbers
  rulecolor=\color{black},         % if not set, the frame-color may be changed on line-breaks within not-black text (e.g. comments (green here))
  showspaces=false,                % show spaces everywhere adding particular underscores; it overrides 'showstringspaces'
  showstringspaces=false,          % underline spaces within strings only
  showtabs=false,                  % show tabs within strings adding particular underscores
  stepnumber=2,                    % the step between two line-numbers. If it's 1, each line will be numbered
  stringstyle=\color{mymauve},     % string literal style
  tabsize=2,	                   % sets default tabsize to 2 spaces
  title=\lstname                   % show the filename of files included with \lstinputlisting; also try caption instead of title
}

\newcommand{\matlab}{\textsc{Matlab}}

% Math symbols
\usepackage{amsmath}
\usepackage{amssymb}
\usepackage{amsthm}
\DeclareMathOperator*{\argmin}{arg\,min}
\DeclareMathOperator*{\argmax}{arg\,max}


% Sets
\newcommand{\Z}{\mathbb{Z}}
\newcommand{\R}{\mathbb{R}}
\newcommand{\Rn}{\R^n}
\newcommand{\Rnn}{\R^{n \times n}}
\newcommand{\C}{\mathbb{C}}
\newcommand{\K}{\mathbb{K}}
\newcommand{\Kn}{\K^n}
\newcommand{\Knn}{\K^{n \times n}}

% Unit vectors
\usepackage{esint}
\usepackage{esvect}
\newcommand{\kmath}{k}
\newcommand{\xunit}{\hat{\imath}}
\newcommand{\yunit}{\hat{\jmath}}
\newcommand{\zunit}{\hat{\kmath}}
\newcommand{\uunit}{\hat{\umath}}

% rot & div & grad & lap
\DeclareMathOperator{\newdiv}{div}
\newcommand{\divn}[1]{\nabla \cdot #1}
\newcommand{\rotn}[1]{\nabla \times #1}
\newcommand{\grad}[1]{\nabla #1}
\newcommand{\gradn}[1]{\nabla #1}
\newcommand{\lap}[1]{\nabla^2 #1}


% Elec
\newcommand{\B}{\vec B}
\newcommand{\E}{\vec E}
\newcommand{\EMF}{\mathcal{E}}
\newcommand{\perm}{\varepsilon} % permittivity

\newcommand{\bigoh}{\mathcal{O}}
\newcommand\eqdef{\triangleq}

\DeclareMathOperator{\newdiff}{d} % use \dif instead
\newcommand{\dif}{\newdiff\!}
\newcommand{\fpart}[2]{\frac{\partial #1}{\partial #2}}
\newcommand{\ffpart}[2]{\frac{\partial^2 #1}{\partial #2^2}}
\newcommand{\fdpart}[3]{\frac{\partial^2 #1}{\partial #2\partial #3}}
\newcommand{\fdif}[2]{\frac{\dif #1}{\dif #2}}
\newcommand{\ffdif}[2]{\frac{\dif^2 #1}{\dif #2^2}}
\newcommand{\constant}{\ensuremath{\mathrm{cst}}}

%\KOMAoptions{DIV=last}

\usepackage{circuitikz}
\usetikzlibrary{babel}
\ctikzset{tripoles/mos style/arrows}

\makeatletter
\pgfcircdeclaremos{pmos}{
          \anchor{S}{
            \northeast
          }
          \anchor{source}{
            \northeast
          }
          \anchor{D}{
            \northeast
            \pgf@y=-\pgf@y
          }
          \anchor{drain}{
            \northeast
            \pgf@y=-\pgf@y
          }
}{%
            \pgfpathmoveto{\pgfpoint{\pgf@circ@res@right}{\pgf@circ@res@up}}
            \pgfpathlineto{\pgfpoint{\pgf@circ@res@right}
                {\pgfkeysvalueof{/tikz/circuitikz/tripoles/pmos/gate height}\pgf@circ@res@up}}
            \pgfpathlineto{\pgfpoint
                {\pgfkeysvalueof{/tikz/circuitikz/tripoles/pmos/base width}\pgf@circ@res@left}
                {\pgfkeysvalueof{/tikz/circuitikz/tripoles/pmos/gate height}\pgf@circ@res@up}}
            \pgfusepath{draw}

        \ifpgf@circuit@mos@arrows
            \pgfscope             
            \pgfslopedattimetrue 
            \pgfallowupsidedownattimetrue
            \pgfresetnontranslationattimefalse
            \pgftransformlineattime{.4}{%
                \pgfpoint%
                    {\pgf@circ@res@right}%
                    {\pgfkeysvalueof{/tikz/circuitikz/tripoles/pmos/gate height}\pgf@circ@res@up}%
            }{%
                \pgfpoint
                    {\pgfkeysvalueof{/tikz/circuitikz/tripoles/pmos/gate width}\pgf@circ@res@left}%
                    {\pgfkeysvalueof{/tikz/circuitikz/tripoles/pmos/gate height}\pgf@circ@res@up}%
            }
            \pgfnode{currarrow}{center}{}{}{\pgfusepath{stroke}}
            \endpgfscope
        \fi

            \pgfscope
            \pgfpathmoveto{\pgfpoint
                {\pgfkeysvalueof{/tikz/circuitikz/tripoles/pmos/base width}\pgf@circ@res@left}
                {\pgfkeysvalueof{/tikz/circuitikz/tripoles/pmos/base height}\pgf@circ@res@up}}
            \pgfpathlineto{\pgfpoint
                {\pgfkeysvalueof{/tikz/circuitikz/tripoles/pmos/base width}\pgf@circ@res@left}
                {\pgfkeysvalueof{/tikz/circuitikz/tripoles/pmos/base height}\pgf@circ@res@down}}
            \pgfsetlinewidth{2\pgflinewidth}
            \pgfusepath{draw}
            \endpgfscope

            \pgfpathmoveto{\pgfpoint
                {\pgfkeysvalueof{/tikz/circuitikz/tripoles/pmos/base width}\pgf@circ@res@left}
                {\pgfkeysvalueof{/tikz/circuitikz/tripoles/pmos/gate height}\pgf@circ@res@down}}
            \pgfpathlineto{\pgfpoint{\pgf@circ@res@right}
                {\pgfkeysvalueof{/tikz/circuitikz/tripoles/pmos/gate height}\pgf@circ@res@down}}
            \pgfpathlineto{\pgfpoint{\pgf@circ@res@right}{\pgf@circ@res@down}}      

            \pgfpathmoveto{\pgfpoint
                {\pgfkeysvalueof{/tikz/circuitikz/tripoles/pmos/gate width}\pgf@circ@res@left}
                {\pgfkeysvalueof{/tikz/circuitikz/tripoles/pmos/gate height}\pgf@circ@res@up}}
            \pgfpathlineto{\pgfpoint
                {\pgfkeysvalueof{/tikz/circuitikz/tripoles/pmos/gate width}\pgf@circ@res@left}
                {\pgfkeysvalueof{/tikz/circuitikz/tripoles/pmos/gate height}\pgf@circ@res@down}}


            \pgfpathmoveto{\pgfpoint
                {\pgfkeysvalueof{/tikz/circuitikz/tripoles/pmos/gate width}\pgf@circ@res@left}
                {\pgf@circ@res@up+\pgf@circ@res@down}}
            \pgfpathlineto{\pgfpoint{\pgf@circ@res@left}{\pgf@circ@res@up+\pgf@circ@res@down}}
            \pgfusepath{draw}

%           \pgfpathcircle{\pgfpoint
%               {\pgfkeysvalueof{/tikz/circuitikz/tripoles/pmos/gate width}\pgf@circ@res@left - \pgfkeysvalueof{/tikz/circuitikz/nodes width}*\pgfkeysvalueof{/tikz/circuitikz/bipoles/length}}
%               {\pgf@circ@res@up+\pgf@circ@res@down}}{\pgfkeysvalueof{/tikz/circuitikz/nodes width}*\pgfkeysvalueof{/tikz/circuitikz/bipoles/length}}
%           \pgfusepath{draw,fill}      

}
\usetikzlibrary{intersections}
\usepgfplotslibrary{fillbetween}
\pgfdeclarelayer{bg}
\pgfsetlayers{bg,main}

\titlehead{}
\subject{LELEC1530}
\title{Séance complémentaire - circuits digitaux}
\subtitle{Solutions}
\author{\small Gaëtan \textsc{Cassiers} \and\small Antoine \textsc{Paris}}
\date{}

\begin{document}
\maketitle

\emph{Cette séance est basée sur la partie "exercices complémentaires" du syllabus d'exercice.}

\section*{Exercice 4 : Circuit combinatoire}
Implémentez et dimensionnez la fonction $X = \overline{ABCD} + E$ en logique domino
et en CMOS sur base de l'inverseur unitaire ($L_{min}=\SI{50}{nm}$, $W_n=\SI{500}{nm}$,
$W_p=\SI{1000}{nm}$). Comparez la surface de silicium nécessaire à l'implémentation en
ne comptant que la surface occupée par les transistors.

\hspace{1cm}\hrule\hspace{1cm}

\subsection*{Implémentation en logique CMOS}
En logique CMOS, une porte logique est constituée d'une partie \emph{pull-up} (en PMOS, car
les PMOS sont bons pour passer des 1) et d'une partie \emph{pull-down} (en NMOS, car les
NMOS sont bons pour passer des 0). La sortie $X$ d'un circuit en logique CMOS peut se
trouver de deux manières :
\begin{enumerate}
	\item En passant par le réseau pull-up : $X = f_u(\bar{A}, \bar{B}, \dots)$ ;
	\item En passant par le réseau pull-down : $X = \overline{f_d(A, B, \dots)}$.
\end{enumerate}
En logique CMOS, on a toujours $f_u(\bar{A}, \bar{B}, \dots) = \overline{f_d(A, B, \dots)}$.
Cela signifie deux choses. Premièrement, la sortie n'est jamais tirée à la fois vers la 
masse et vers $V_{DD}$. Deuxièmement la sortie n'est jamais flottante (elle est toujours
tirée soit vers la masse, soit vers $V_{DD}$). \\

Dans cet exercice, $X = \overline{ABCD} + E$. On constate donc qu'il est impossible
d'écrire $X$ comme une fonction uniquement des compléments $\bar{A}, \bar{B}, \dots$
ou comme le complément d'une fonction des litéraux $A, B, \dots$. On a donc deux
possibilités :
\begin{enumerate}
	\item On peut réecrire $X = \overline{ABCD\bar{E}}$. La fonction logique peut
	donc être implémentée en un seul étage avec une NAND à 5 entrées et un inverseur
	pour obtenir $\bar{E}$ ;
	\item La fonction logique peut être implémentée en plusieurs étages : une NAND
	à 4 entrée pour obtenir $\overline{ABCD}$, suivi d'une NOR et puis d'une NOT pour
	obtenir $X$.
\end{enumerate}

Si le but est d'optimiser la surface de silicium utilisée, la deuxième solution
est légèrement préférable. L'implémentation est donnée à la figure~\ref{fig:ex4-cmos}.

\begin{figure}
	\centering
	\begin{circuitikz}
		\draw
		% 4NAND
		(2.25,2) node[vcc] {$V_{DD}$}
		(0,2) -- (4.5,2)
    	(0,0) to [Tpmos,n=p1] (0,2)
    	(p1.G) node[anchor=east] {$A$}
    	(1.5,0) to [Tpmos, n=p2] (1.5,2)
    	(p2.G) node[anchor=east] {$B$}
    	(3.0,0) to [Tpmos, n=p3] (3.0,2)
    	(p3.G) node[anchor=east] {$C$}
    	(4.5,0) to [Tpmos, n=p4] (4.5,2)
    	(p4.G) node[anchor=east] {$D$}
    	(0,0) -- (6,0)
    	(2.25, -1.5) to [Tnmos, n=n1] (2.25, 0)
    	(n1.G) node[anchor=east] {$A$}
    	(2.25, -3.0) to [Tnmos, n=n2] (2.25, -1.5)
    	(n2.G) node[anchor=east] {$B$}
    	(2.25, -4.5) to [Tnmos, n=n3] (2.25, -3.0)
    	(n3.G) node[anchor=east] {$C$}
    	(2.25, -6) to [Tnmos, n=n4] (2.25, -4.5)
    	(n4.G) node[anchor=east] {$D$}
    	(2.25, -6) node[ground] {}
    	
    	% 2NOR
    	(8,2) node[vcc] {$V_{DD}$}
    	(8, 0.5) to [Tpmos, n=p5] (8,2)
    	(6,0) -- (6,1.25) -- (p5.G)
    	(8, -1) to [Tpmos, n=p6] (8,0.5)
    	(p6.G) node[anchor=east] {$E$}
    	(7.25,-2.5) to [Tnmos,n=n5] (7.25,-1)
    	(8.75,-2.5) to [Tnmos, n=n6] (8.75,-1)
    	(n6.G) node[anchor=east] {$E$}
    	(6,0) -- (6,-1.75) -- (n5.G)
    	(7.25, -1) -- (8.75, -1)
    	(7.25, -2.5) -- (8.75, -2.5)
    	(8, -2.5) node[ground] {}
    	
    	% NOT
    	(10, -1) node[not port](mynot){}
    	(8.75, -1) -- (mynot.in)
    	(mynot.out) node[anchor=west] {$X$}
    	;
	\end{circuitikz}
	\caption{Implémentation CMOS de $X = \overline{ABCD} + E$.}
	\label{fig:ex4-cmos}
\end{figure}

\paragraph{Dimensionnement}
On dimensionne ensuite comme d'habitude par rapport à l'inverseur unitaire.
\begin{enumerate}
	\item PMOS de la 4-NAND : $20/1$ ;
	\item NMOS de la 4-NAND : $40/1$ ;
	\item PMOS de la 2-NOR : $40/1$ ;
	\item NMOS de la 2-NOR : $10/1$. 
\end{enumerate}
On trouve donc que la surface totate est donnée par $370L_{min}^2$ (l'autre solution
proposée donne une surface de $380L_{min}^2$).

\subsection*{Implémentation en logique domino}
Une implémentation possible est donnée à la figure~\ref{fig:ex4-domino}.
Il est impossible, en logique domino, d'implémenter une fonction d'inversion
sans disposer des compléments des entrées. On suppose donc ici que l'on dispose
des ces compléments (dans tout les cas, on peut les obtenir via un inverseur).

\begin{figure}
	\centering
	\begin{circuitikz}
		\draw
		(3,2) to[Tpmos, n=pl] (3,4)
		(pl.S) node[vcc] {$V_{DD}$}
		(pl.G) node[anchor=east] {$CLK$}
		(0,2) -- (6,2)
		(0,0) to[Tnmos, n=na] (0,2)
		(na.G) node[anchor=east] {$\bar{A}$}
		(1.5,0) to[Tnmos, n=nb] (1.5,2)
		(nb.G) node[anchor=east] {$\bar{B}$}
		(3.0,0) to[Tnmos, n=nc] (3.0,2)
		(nc.G) node[anchor=east] {$\bar{C}$}
		(4.5,0) to[Tnmos, n=nd] (4.5,2)
		(nd.G) node[anchor=east] {$\bar{D}$}
		(6.0,0) to[Tnmos, n=ne] (6.0,2)
		(ne.G) node[anchor=east] {$E$}
		(0,0) -- (6,0)
		(3,-2) to[Tnmos, n=pev] (nc.S)
		(pev.G) node[anchor=east] {$CLK$}
		(pev.S) node[ground] {}
		
		(7,2) node[not port](mynot) {}
		(6,2) -- (mynot.in)
    	(mynot.out) node[anchor=west] {$X$}
    	;
	\end{circuitikz}
	\caption{Implémentation en logique domino de $X = \overline{ABCD} + E$.}
	\label{fig:ex4-domino}
\end{figure}

Une autre implémentation est possible en faisant une AND de $A$, $B$, $C$, $D$
et $\bar{E}$. On pourrait être tenter de dire que cette implémentation occupera
moins de surface de silicium que celle présentée à la figure~\ref{fig:ex4-domino}
car on utilise 3 inverseurs de moins. C'est sans compter sur le fait que l'implémentation
AND compte 6 NMOS en série (5 NMOS avec $A$, $B$, $C$, $D$, $\bar{E}$ et le NMOS
d'évaluation). Or on sait que, après dimensionnement par rapport à l'inverseur unitaire,
des NMOS en série sont bien moins avantageux que des NMOS en parallèle en terme
de surface.

\paragraph{Dimensionnement}
On dimensionne comme d'habitude par rapport à l'inverseur unitaire.
\begin{itemize}
	\item PMOS de pré-charge : $20/1$ ;
	\item NMOS : $20/1$.
\end{itemize}
On a donc une surface totale de $290L_{min}^2$ (en comptant les 4 inverseurs
pour obtenir les compléments des entrées), soit 22\% de moins qu'en logique
CMOS.

\clearpage
\section*{Exercice 5: Logique dynamique}
On considère le circuit en logique dynamique de la figure~\ref{fig9-2} dans une technologie
\SI{50}{nm}. Voici les paramètres technologiques :
\begin{center} 
	$\begin{array}{l l l l}
		L_{min} = \SI{50}{nm} & C_{ox} = 62.5\cdot WL\text{aF} 
		& R_N= 34/W k\Omega
		& R_P= 68/W k\Omega \\ 
	\end{array}$
\end{center}

\begin{figure}[]
   \centering
   \includegraphics[width=14cm]{figures/fig9-2.png}
   \caption{Exercice 5, logique dynamique.}
   \label{fig9-2}
\end{figure}

\begin{enumerate}
	\item Quelle est la fonction logique implémentée ?
	\item Dimensionnez la porte par rapport à l'inverseur unitaire ($L_{min}=\SI{50}{nm}$,
	$W_n=\SI{500}{nm}$ et $W_p=\SI{1000}{nm}$).
	\item Calculez le délai $t_{PHL}$ à 50\% lorsque $A_0$=1 et $A_{1->4}$=0 si la sortie
	est connectée à une capacité de charge $C_L$ de \SI{50}{\femto\farad}.
\end{enumerate}

\hspace{1cm}\hrule\hspace{1cm}

\paragraph{Fonction logique implémentée}
La fonction logique implémentée est donnée par
\[ \text{Out} = \overline{A_0 + A_1A_2 + A_3A_4}.\]

\paragraph{Dimensionnement}
On dimensionne comme d'habitude par rapport à l'inverseur unitaire.
\begin{itemize}
	\item PMOS de pré-charge : $20/1$ ;
	\item NMOS $A_{1\to4}$ et NMOS d'évaluation : $30/1$ ;
	\item NMOS $A_0$ : $15/1$. 
\end{itemize}

\paragraph{Délai}
Si $A_{1\to4} = 0$, la sortie se décharge uniquement à travers $A_0$
et le NMOS d'évaluation. On a donc une décharge à travers deux transistors
en série. On sait que le temps de déchage $t_{PHL}$ à travers $N$
transistors \emph{identiques} en séries peut être estimé par
\[ t_{PHL} = 0.35R_nC_{oxn}N^2 + 0.7NR_nC_{load}. \]
Le problème ici est que le transistor $A_0$ est différent du transistor
d'évaluation vu le dimensionnement effectué au point précédent.
Une approche serait par exemple d'utiliser la moyenne de $R_n$ et $C_{oxn}$
sur ces deux transistors. Cela donne
\[ R_{n,avg} = \frac{1}{2} \cdot \left(\frac{34}{15} + \frac{34}{30}\right)\SI{}{\kilo\ohm}
= \SI{1.7}{\kilo\ohm} \]
et
\[ C_{oxn,avg} = \frac{1}{2} \cdot \left(62.5\cdot 15 + 62.5\cdot 30)\right)\SI{}{\atto\farad}
= \SI{2.9125}{\femto\farad}. \]
Quand à $C_{load}$, celle ci comprend la capacité de charge $C_L$ ainsi que
les capacités parasites des transistors. Il est utile ici de rappeller
quel modèle de transistor utiliser. Lors de la première séance d'exercices sur
le MOS en digital, nous avons utilisé le modèle de droite dans la figure~\ref{fig:mos-model}
pour dériver les formules approximatives de $t_{PHL}$ et $t_{PLH}$.
Celui-ci a été obtenu du modèle de gauche dans cette même figure en utilisant
l'effet Miller que l'on peut appliquer lors d'une transition à l'entrée en $G$.
Il était donc approprié d'utiliser ce modèle dans la première séance puisqu'on
avait effectivement une transition à l'entrée.
Dans le circuit de la figure~\ref{fig9-2}, les transistors $A_{1\to 4}$ sont
bloqués et il n'y a pas de transitions aux entrées. Utiliser le modèle de droite
de la figure~\ref{fig:mos-model} n'a donc pas de ce sens puisque celui-ci a été
obtenu dans le cas d'une transition à la grille. Pour prendre en compte les capacités
parasites, il faut donc utiliser le modèle de gauche dans la figure~\ref{fig:mos-model}
pour les transistors $A_{1\to 4}$.\\

\begin{figure}
	\centering
	\begin{tikzpicture}
		\draw
		(-3, 0) to[Tnmos, n=n] (-3,2)
		(n.G) --++(-0.5,0) to[C=$\frac{1}{2}C_{ox}$] ++(0,1) -- (-3,2)
		(n.G) --++(-0.5,0) to[C,l_=$\frac{1}{2}C_{ox}$] ++(0,-1) -- (-3,0)
		(n.G) node[anchor=east,xshift=-0.6cm] {$G$}
		(n.D) node[anchor=west] {$D$}
		(n.S) node[anchor=west] {$S$}	
		
    	(0.5, 2) node[anchor=east] {$G$} to [short, o-] (1, 2)
    	(1, 2) to [C=$\frac{3}{2}C_{ox}$] (1, 0)
    	(1, 0) -- (4, 0)
    	(4, 0) to [R=$R_n$] (4, 2)
    	(4, 2) to [switch] (6,2)
    	(6, 2) to [C=$C_{ox}$] (6,0) -- (4, 0)
    	(6, 2) to [short, -o] (6.5,2) node[anchor=west] {$D$}
    	(4, 0) to [short, -o] (4,-0.5) node[anchor=north] {$S$};
    	
    	\draw [->] (-2,1) -- node[text width=1cm,midway,above] {Miller} (-0.5,1);
	\end{tikzpicture}
	\caption{A gauche, le modèle \emph{général} d'un transistor
	MOSFET avec ses capacités parasites. A droite, ce même modèle
	transformé via Miller lors d'une \emph{transition} à l'entrée $G$.}
	\label{fig:mos-model}
\end{figure}

$C_{load}$ est donc composé de $C_L$, la capacité
du PMOS d'évaluation $C_{ox,p}/2$ et les capacités $C_{ox,n}/2$ des NMOS
$A_1$ et $A_3$ qui sont en parallèle. Il ne faut pas compter la capacité
du transistor par lequel se décharge le noeud (c'est à dire $A_0$).
On a donc
\[ C_{load} = C_L + \frac{C_{ox,p}}{2} + C_{ox,n} = \SI{50}{\femto\farad}
+ \frac{1}{2}\cdot\SI{1.25}{\femto\farad} + \SI{1.875}{\femto\farad} = \SI{52.5}{\femto\farad}. \].
En pratique toute fois, on a souvent $C_L \gg \frac{C_{ox,p}}{2} + C_{ox,n}$
et donc $C_{load} \approx C_L$.  Dans notre cas, cela permet de simplifier
le calcul en ne commettant qu'une faible erreur (ici 5\%, comme
la formule qu'on utilise n'est quand même précise qu'à un facteur 2, ce n'est
pas très grave). On trouve donc un délai
\[ t_{PHL} = \SI{6.93}{\pico\second} + \SI{124.95}{\pico\second} =
\SI{131.88}{\pico\second} \]
sans faire d'approximation et de
\[ t_{PHL} \approx \SI{6.93}{\pico\second} + \SI{119}{\pico\second}
= \SI{125.9}{\pico\second} \]
avec $C_{load} \approx C_L$.
On voit que le terme en $N^2$ aurait également pû être négligé sans une grande
perte de précision.

\clearpage
\section*{Exercice 6: full adder}

On considère un additionneur sur 1 bit présenté à la figure~\ref{fig9-3} dans
une technologie \SI{50}{nm}. Voici les paramètres technologiques :
\begin{align*}
    L_{min} &= \SI{50}{nm} &
    C_{ox} &= \SI{62.5}{aF}\cdot W\cdot L &
    R_N&=\SI{34}{k\ohm}/W &
    R_P&=\SI{68}{k\ohm}/W 
\end{align*}

\begin{figure}
	\centering
	\includegraphics[width=13cm]{figures/fig9-3.png}
	\caption{Exercice 6, full adder.}
	\label{fig9-3}
\end{figure}

\begin{enumerate}
	\item Décrivez le type de logique utilisé dans la Figure \ref{fig9-3}.
	Donnez la fonctionnalité logique de chacune des deux portes sous forme d'une
	table de vérité ainsi que la fonctionnalité logique globale.
	\item Dimensionnez les deux portes logique sur base de l'inverseur
    unitaire ($L_{min}=\SI{50}{nm}$, $W_{n}=\SI{500}{nm}$, $W_{p}=\SI{1000}{nm}$).
	\item Implémentez un additionneur prenant en entrée des mots de 2 bits sur
	base de l'additionneur sur 1 bit de la Figure \ref{fig9-3}.
	\item En considérant un signal d'horologe avec un \emph{duty-cycle} de 50\%,
	quelle est la fréquence maximale de fonctionnement de l'additionneur 2 bits ?
	Quelle est la fréquence maximale si l'architecture implémente une addition sur
	2 mots de 32 bits ?
\end{enumerate}

\hspace{1cm}\hrule

\subsection*{1. Type de logique, équations et table de vérité}

Logique domino NP (Zipper logic).

\paragraph{Équations logiques}
$C_{n+1} = A_n B_n + C_n (A_n + B_n)$
\begin{align*}
    S_n &= \overline{(\overline{A_n} + \overline{B_n} + \overline{C_n})(C_{n+1} + \overline{A_n}\cdot\overline{B_n}\cdot\overline{C_n})} \\
    &=  \overline{\overline{A_n}+\overline{B_n}+\overline{C_n}} + \overline{C_{n+1} + \overline{A_n}\cdot\overline{B_n}\cdot\overline{C_n}} \\
    &= A_n B_n C_n + \overline{C_{n+1}}(A_n+B_n+C_n)
\end{align*}

\paragraph{Table de vérité}

\begin{center}
    \begin{tabular}{ccc|cc}
        $A_n$&$B_n$&$C_n$&$C_{n+1}$&$S_n$\\
        \hline
        0&0&0&0&0\\
        0&0&1&0&1\\
        0&1&0&0&1\\
        0&1&1&1&0\\
        1&0&0&0&1\\
        1&0&1&1&0\\
        1&1&0&1&0\\
        1&1&1&1&1
    \end{tabular}
\end{center}

\subsection*{2. Dimensionnement}

Largeur de transistors (relative à la largeur du transistor correspondant
dans l'inverseur unitaire.

Calcul du \emph{carry} (circuit de gauche):
\begin{itemize}
    \item tous les NMOS: 3
    \item PMOS: 1
\end{itemize}

Calcul de la somme (circuit de droite):
\begin{itemize}
    \item NMOS: 1
    \item PMOS phi: 5
    \item PMOS An, Bn, Cn: 5
    \item PMOS $C_{n+1}$: 3
\end{itemize}

\subsection*{3. Additionneur 2-bit}

Utiliser deux full adders 1 bit ($FA_0$ et $FA_1$), connecter le 
\emph{carry-out} ($C_1$) du premier adder au \emph{carry-in} du deuxième. 

\subsection*{4. Fréquence maximale}

\paragraph{Calcul du délai}

On considère le délai pour un additionneur 1 bit.
Pour un additionneur, on peut considérer 3 délais:
\begin{itemize}
    \item Délai pour le calcul de $\overline{C_{n+1}}$: $t_0$
    \item Délai pour le calcul de $C_{n+1}$ (à partir de $\overline{C_{n+1}}$): $t_1$
    \item Délai pour le calcul de $\overline{S_n}$ (à partir de $\overline{C_{n+1}}$): $t_2$
    \item Délai pour le calcul de $S_n$ (à partur de $\overline{S_n}$): $t_3$
\end{itemize}

On note $C_{ox} = C_{ox,n} = \SI{625}{aF}$ (capacité pour le NMOS $10/1$ 
d'un inverseur unitaire). On a également $C_{ox,p} = 2C_{ox,n} = 2C_{ox}$ (capacité
pour le PMOS $20/1$ d'un inverseur unitaire). Tout sera
donc calculé directement en fonction de $C_{ox}$ dans la suite de l'exercice.\\

Au niveau des modèles à utiliser pour chaque transistor, on suppose que les
entrées $A_n$, $B_n$ et $C_n$ sont fixes lorsque qu'on passe de la phase de
pré-charge à la phase d'évaluation. On utilise donc le modèle de gauche de la
figure~\ref{fig:mos-model} pour ces transistors. Pour les autres, il y a une
transitions à la grille, on utilise donc le modèle de droite de la figure~\ref{fig:mos-model}. 

\subparagraph{Calcul de $t_0$}
Dans le pire cas, chaine de 3 NMOS de largeur 3.
Capacité du noeud $\overline{C_{n+1}}$:
\begin{itemize}
    \item Sortie des NMOS: $3/2 C_{ox}$ (on prend en compte un seul NMOS, l'autre est
    déjà pris en compte par le terme en $N^2$ de la formule du délai)
    \item PMOS (de precharge): $2C_{ox}$ (car effet Miller)
    \item Entrée de l'inverseur: $9/2 C_{ox}$
    \item Entrée du PMOS du circuit de la somme: $9 C_{ox}$ (car effet Miller)
\end{itemize}

On a donc les paramètres suivants:
\begin{itemize}
    \item Capacité totale de sortie: $C_{out0} = 17 C_{ox0}$.
    \item Capacité de grille des NMOS: $3C_{ox}$
    \item Résistance des NMOS: $R_N/30$
    \item Nombre de NMOS: $N = 3$
\end{itemize}

On a donc
\[t_0 = 0.35 N^2 (3C_{ox}) R_N/30 + 0.7 C_{out0} N R_N/30 = \SI{32}{ps}\]

\subparagraph{Calcul de $t_1$}
Capacité du noeud de sortie $C_{n+1}$: on considère que le carry-out est
connécté au carry-in d'un full adder.
\begin{itemize}
\item Capacité de sortie de l'inverseur $3C_{ox}$
\item NMOS: $9/2 C_{ox}$ (largeur 3, effet Miller)
\item PMOS: $2\cdot 5 \cdot 2 \ cdot 3/2 C_{ox} = 30 C_{ox}$ (deux PMOS de largeur 5 avec effet Miller)
\end{itemize}
Capacité totale: $C_{out1} = 70.5 C_{ox}$.

On a un inverseur, donc la résistance équivalente est $R_N/10$. On a donc
\[t_1 = 0.7 \cdot R_N/10 C_{out1} = \SI{105}{ps}\]

\subparagraph{Calcul de $t_2$}
Dans le pire cas, chaine de 5 PMOS de largeur 5.
Capacité du noeud $\overline{S_n}$:
\begin{itemize}
    \item NMOS: $C_{ox}$ (car effet Miller)
    \item PMOS ($\overline{C_{n+1}}$): $3C_{ox}$ (un seul PMOS de largeur 3)
    \item Entrée de l'inverseur: $9/2 C_{ox}$
\end{itemize}

On a donc les paramètres suivants:
\begin{itemize}
    \item Capacité totale de sortie: $C_{out2} = 9 C_{ox}$.
    \item Capacité de grille des PMOS: $10 C_{ox}$
    \item Résistance des PMOS: $R_P/100$
    \item Nombre de PMOS: $N = 5$
\end{itemize}
On a donc
\[t_2 = 0.35 N^2 (10C_{ox}) R_P/100 + 0.7 C_{out2} N R_P/100 = \SI{51}{ps}\]

\subparagraph{Calcul de $t_3$}
Sortie d'un inverseur:
\[t_3 = 0.7 \cdot 3 \cdot C_{ox} R_N/10 = \SI{4.5}{ps}\]

\subparagraph{Délai total}
Au total, le délai pour un additionneur de 2 bits est
\[t_{FA2} = t_0 + t_1 + t_0 + t_2 + t_3 = \SI{224}{ps}\]

\paragraph{Fréquence maximale}
Comme seulement la moitié de le période d'horloge est disponible pour
effectuer les calculs (l'autre moitié est utilisée pour le \emph{prefetch}),
la période d'horloge minimale est \SI{448}{ps}. La fréquence maximale est
donc \SI{2.2}{GHz}.

\paragraph{Additionneur de 32 bits}
Le délai est
\[t_{F32} = 31 (t_0 + t_1) + t_0 + t_2 + t_3 = \SI{4.3}{ns},\]
donc la période d'horloge minimale est de \SI{8.7}{ns}, donc la
fréquence maximale est \SI{115}{MHz}.
\end{document}

