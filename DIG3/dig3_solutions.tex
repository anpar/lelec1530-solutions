\documentclass[frenchb,DIV=14]{scrartcl}

\usepackage[utf8x]{inputenc}
\usepackage[T1]{fontenc}
\usepackage{lmodern}
\usepackage{microtype}
\usepackage{xspace}
\usepackage[binary-units=true]{siunitx}
%\usepackage{color}
%\usepackage[usenames,dvipsnames,svgnames,table]{xcolor}
\usepackage{graphicx}
\usepackage{hyperref}
\usepackage{todonotes}
\usepackage{epstopdf}
\usepackage{array}
\usepackage{multicol}
\usepackage{multirow}
\usepackage{tabularx} % tabular with automatic line-break
\newcolumntype{Y}{>{\centering\arraybackslash}X} % centered column
\usepackage{ifthen}
\usepackage{amsmath}
\usepackage{grffile} % better name handling with graphicx
\usepackage{currfile} % provides relative file inclusion for tikzscale

\usepackage{tikz}
\usepackage{pgfplots}
\usepackage{tikz-3dplot}
\usepackage{pgfplots}
\usepackage{tikzscale}
\pgfplotsset{compat=newest}
\usetikzlibrary{plotmarks}
\usepackage{rotating}

\usepackage[french]{babel}
% Listing
% always put it after babel
% http://tex.stackexchange.com/questions/100717/code-in-lstlisting-breaks-document-compile-error
\usepackage{listings}

\definecolor{mygreen}{rgb}{0,0.6,0}
\definecolor{mygray}{rgb}{0.5,0.5,0.5}
\definecolor{mymauve}{rgb}{0.58,0,0.82}
\lstset{ %
  language=Matlab,
  backgroundcolor=\color{white},   % choose the background color; you must add \usepackage{color} or \usepackage{xcolor}
  basicstyle=\footnotesize,        % the size of the fonts that are used for the code
  breakatwhitespace=false,         % sets if automatic breaks should only happen at whitespace
  breaklines=true,                 % sets automatic line breaking
  captionpos=b,                    % sets the caption-position to bottom
  commentstyle=\color{mygreen},    % comment style
  deletekeywords={...},            % if you want to delete keywords from the given language
  escapeinside={\%*}{*)},          % if you want to add LaTeX within your code
  extendedchars=true,              % lets you use non-ASCII characters; for 8-bits encodings only, does not work with UTF-8
  frame=single,	                   % adds a frame around the code
  keepspaces=true,                 % keeps spaces in text, useful for keeping indentation of code (possibly needs columns=flexible)
  keywordstyle=\color{blue},       % keyword style
  otherkeywords={*,...},           % if you want to add more keywords to the set
  numbers=none,                    % where to put the line-numbers; possible values are (none, left, right)
  numbersep=5pt,                   % how far the line-numbers are from the code
  numberstyle=\tiny\color{mygray}, % the style that is used for the line-numbers
  rulecolor=\color{black},         % if not set, the frame-color may be changed on line-breaks within not-black text (e.g. comments (green here))
  showspaces=false,                % show spaces everywhere adding particular underscores; it overrides 'showstringspaces'
  showstringspaces=false,          % underline spaces within strings only
  showtabs=false,                  % show tabs within strings adding particular underscores
  stepnumber=2,                    % the step between two line-numbers. If it's 1, each line will be numbered
  stringstyle=\color{mymauve},     % string literal style
  tabsize=2,	                   % sets default tabsize to 2 spaces
  title=\lstname                   % show the filename of files included with \lstinputlisting; also try caption instead of title
}

\newcommand{\matlab}{\textsc{Matlab}}

% Math symbols
\usepackage{amsmath}
\usepackage{amssymb}
\usepackage{amsthm}
\DeclareMathOperator*{\argmin}{arg\,min}
\DeclareMathOperator*{\argmax}{arg\,max}


% Sets
\newcommand{\Z}{\mathbb{Z}}
\newcommand{\R}{\mathbb{R}}
\newcommand{\Rn}{\R^n}
\newcommand{\Rnn}{\R^{n \times n}}
\newcommand{\C}{\mathbb{C}}
\newcommand{\K}{\mathbb{K}}
\newcommand{\Kn}{\K^n}
\newcommand{\Knn}{\K^{n \times n}}

% Unit vectors
\usepackage{esint}
\usepackage{esvect}
\newcommand{\kmath}{k}
\newcommand{\xunit}{\hat{\imath}}
\newcommand{\yunit}{\hat{\jmath}}
\newcommand{\zunit}{\hat{\kmath}}
\newcommand{\uunit}{\hat{\umath}}

% rot & div & grad & lap
\DeclareMathOperator{\newdiv}{div}
\newcommand{\divn}[1]{\nabla \cdot #1}
\newcommand{\rotn}[1]{\nabla \times #1}
\newcommand{\grad}[1]{\nabla #1}
\newcommand{\gradn}[1]{\nabla #1}
\newcommand{\lap}[1]{\nabla^2 #1}


% Elec
\newcommand{\B}{\vec B}
\newcommand{\E}{\vec E}
\newcommand{\EMF}{\mathcal{E}}
\newcommand{\perm}{\varepsilon} % permittivity

\newcommand{\bigoh}{\mathcal{O}}
\newcommand\eqdef{\triangleq}

\DeclareMathOperator{\newdiff}{d} % use \dif instead
\newcommand{\dif}{\newdiff\!}
\newcommand{\fpart}[2]{\frac{\partial #1}{\partial #2}}
\newcommand{\ffpart}[2]{\frac{\partial^2 #1}{\partial #2^2}}
\newcommand{\fdpart}[3]{\frac{\partial^2 #1}{\partial #2\partial #3}}
\newcommand{\fdif}[2]{\frac{\dif #1}{\dif #2}}
\newcommand{\ffdif}[2]{\frac{\dif^2 #1}{\dif #2^2}}
\newcommand{\constant}{\ensuremath{\mathrm{cst}}}

%\KOMAoptions{DIV=last}

\usepackage{circuitikz}
\usetikzlibrary{babel}
\ctikzset{tripoles/mos style/arrows}

\makeatletter
\pgfcircdeclaremos{pmos}{
          \anchor{S}{
            \northeast
          }
          \anchor{source}{
            \northeast
          }
          \anchor{D}{
            \northeast
            \pgf@y=-\pgf@y
          }
          \anchor{drain}{
            \northeast
            \pgf@y=-\pgf@y
          }
}{%
            \pgfpathmoveto{\pgfpoint{\pgf@circ@res@right}{\pgf@circ@res@up}}
            \pgfpathlineto{\pgfpoint{\pgf@circ@res@right}
                {\pgfkeysvalueof{/tikz/circuitikz/tripoles/pmos/gate height}\pgf@circ@res@up}}
            \pgfpathlineto{\pgfpoint
                {\pgfkeysvalueof{/tikz/circuitikz/tripoles/pmos/base width}\pgf@circ@res@left}
                {\pgfkeysvalueof{/tikz/circuitikz/tripoles/pmos/gate height}\pgf@circ@res@up}}
            \pgfusepath{draw}

        \ifpgf@circuit@mos@arrows
            \pgfscope             
            \pgfslopedattimetrue 
            \pgfallowupsidedownattimetrue
            \pgfresetnontranslationattimefalse
            \pgftransformlineattime{.4}{%
                \pgfpoint%
                    {\pgf@circ@res@right}%
                    {\pgfkeysvalueof{/tikz/circuitikz/tripoles/pmos/gate height}\pgf@circ@res@up}%
            }{%
                \pgfpoint
                    {\pgfkeysvalueof{/tikz/circuitikz/tripoles/pmos/gate width}\pgf@circ@res@left}%
                    {\pgfkeysvalueof{/tikz/circuitikz/tripoles/pmos/gate height}\pgf@circ@res@up}%
            }
            \pgfnode{currarrow}{center}{}{}{\pgfusepath{stroke}}
            \endpgfscope
        \fi

            \pgfscope
            \pgfpathmoveto{\pgfpoint
                {\pgfkeysvalueof{/tikz/circuitikz/tripoles/pmos/base width}\pgf@circ@res@left}
                {\pgfkeysvalueof{/tikz/circuitikz/tripoles/pmos/base height}\pgf@circ@res@up}}
            \pgfpathlineto{\pgfpoint
                {\pgfkeysvalueof{/tikz/circuitikz/tripoles/pmos/base width}\pgf@circ@res@left}
                {\pgfkeysvalueof{/tikz/circuitikz/tripoles/pmos/base height}\pgf@circ@res@down}}
            \pgfsetlinewidth{2\pgflinewidth}
            \pgfusepath{draw}
            \endpgfscope

            \pgfpathmoveto{\pgfpoint
                {\pgfkeysvalueof{/tikz/circuitikz/tripoles/pmos/base width}\pgf@circ@res@left}
                {\pgfkeysvalueof{/tikz/circuitikz/tripoles/pmos/gate height}\pgf@circ@res@down}}
            \pgfpathlineto{\pgfpoint{\pgf@circ@res@right}
                {\pgfkeysvalueof{/tikz/circuitikz/tripoles/pmos/gate height}\pgf@circ@res@down}}
            \pgfpathlineto{\pgfpoint{\pgf@circ@res@right}{\pgf@circ@res@down}}      

            \pgfpathmoveto{\pgfpoint
                {\pgfkeysvalueof{/tikz/circuitikz/tripoles/pmos/gate width}\pgf@circ@res@left}
                {\pgfkeysvalueof{/tikz/circuitikz/tripoles/pmos/gate height}\pgf@circ@res@up}}
            \pgfpathlineto{\pgfpoint
                {\pgfkeysvalueof{/tikz/circuitikz/tripoles/pmos/gate width}\pgf@circ@res@left}
                {\pgfkeysvalueof{/tikz/circuitikz/tripoles/pmos/gate height}\pgf@circ@res@down}}


            \pgfpathmoveto{\pgfpoint
                {\pgfkeysvalueof{/tikz/circuitikz/tripoles/pmos/gate width}\pgf@circ@res@left}
                {\pgf@circ@res@up+\pgf@circ@res@down}}
            \pgfpathlineto{\pgfpoint{\pgf@circ@res@left}{\pgf@circ@res@up+\pgf@circ@res@down}}
            \pgfusepath{draw}

%           \pgfpathcircle{\pgfpoint
%               {\pgfkeysvalueof{/tikz/circuitikz/tripoles/pmos/gate width}\pgf@circ@res@left - \pgfkeysvalueof{/tikz/circuitikz/nodes width}*\pgfkeysvalueof{/tikz/circuitikz/bipoles/length}}
%               {\pgf@circ@res@up+\pgf@circ@res@down}}{\pgfkeysvalueof{/tikz/circuitikz/nodes width}*\pgfkeysvalueof{/tikz/circuitikz/bipoles/length}}
%           \pgfusepath{draw,fill}      

}
\usetikzlibrary{intersections}
\usepgfplotslibrary{fillbetween}
\pgfdeclarelayer{bg}
\pgfsetlayers{bg,main}

\titlehead{}
\subject{LELEC1530}
\title{Séance complémentaire - circuits digitaux}
\subtitle{Solutions}
\author{\small Gaëtan \textsc{Cassiers} \and\small Antoine \textsc{Paris}}
\date{}

\begin{document}
\maketitle

\emph{Cette séance est basée sur la partie "exercices complémentaires" du syllabus}

\section*{Exercice 6: full adder}

On considère un additionneur sur 1 bit présenté à la figure \ref{fig9-3} dans
une technologie \SI{50}{nm}. Voici les paramètres technologiques :
\begin{align*}
    L_{min} &= \SI{50}{nm} &
    C_{ox} &= \SI{62.5}{aF}\cdot W\cdot L &
    R_N&=\SI{34}{k\ohm}/W &
    R_P&=\SI{68}{k\ohm}/W 
\end{align*}

\begin{figure}
\centering
\includegraphics[width=13cm]{figures/fig9-3.png}
\caption{Exercice 1}
\label{fig9-3}
\end{figure}

\begin{enumerate}
\item  Décrivez le type de logique utilisé dans la Figure \ref{fig9-3}.
Donnez la fonctionnalité logique de chacune des deux portes sous forme d'une
table de vérité ainsi que la fonctionnalité logique globale.
\item  Dimensionnez les deux portes logique sur base de l'inverseur
    unitaire ($L_{min}=\SI{50}{nm}$, $W_{n}=\SI{500}{nm}$, $W_{p}=\SI{1000}{nm}$).
\item  Implémentez un additionneur prenant en entrée des mots de 2 bits sur
base de l'additionneur sur 1 bit de la Figure \ref{fig9-3}.
\item  En considérant un signal d'horologe avec un \emph{duty-cycle} de 50\%,
quelle est la fréquence maximale de fonctionnement de l'additionneur 2 bits ?
Quelle est la fréquence maximale si l'architecture implémente une addition sur 2 mots de 32bits ?
\end{enumerate}
\hspace{1cm}\hrule

\subsection*{1.}

Logique domino NP (Zipper logic).

\paragraph{Équations logiques}
$C_{n+1} = A_n B_n + C_n (A_n + B_n)$
\begin{align*}
    S_n &= \overline{(\overline{A_n} + \overline{B_n} + \overline{C_n})(C_{n+1} + \overline{A_n}\cdot\overline{B_n}\cdot\overline{C_n})} \\
    &=  \overline{\overline{A_n}+\overline{B_n}+\overline{C_n}} + \overline{C_{n+1} + \overline{A_n}\cdot\overline{B_n}\cdot\overline{C_n}} \\
    &= A_n B_n C_n + \overline{C_{n+1}}(A_n+B_n+C_n)
\end{align*}

\paragraph{Table de vérité}

\begin{center}
    \begin{tabular}{ccc|cc}
        $A_n$&$B_n$&$C_n$&$C_{n+1}$&$S_n$\\
        \hline
        0&0&0&0&0\\
        0&0&1&0&1\\
        0&1&0&0&1\\
        0&1&1&1&0\\
        1&0&0&0&1\\
        1&0&1&1&0\\
        1&1&0&1&0\\
        1&1&1&1&1
    \end{tabular}
\end{center}

\subsection*{2.}

Largeur de transistors (relative à la largeur du transistor correspondant
dans l'inverseur unitaire.

Calcul du \emph{carry} (circuit de gauche):
\begin{itemize}
    \item tous les NMOS: 3
    \item PMOS: 1
\end{itemize}

Calcul de la somme (circuit de droite):
\begin{itemize}
    \item NMOS: 1
    \item PMOS phi: 5
    \item PMOS An, Bn, Cn: 5
    \item PMOS $C_{n+1}$: 3
\end{itemize}

\subsection*{3.}

Utiliser deux full adders 1 bit ($FA_0$ et $FA_1$), connecter le 
\emph{carry-out} ($C_1$) du premier adder au \emph{carry-in} du deuxième. 

\subsection*{4.}

\paragraph{Calcul du délai}

On considère le délai pour un additionneur 1 bit.
Pour un additionneur, on peut considérer 3 délais:
\begin{itemize}
    \item Délai pour le calcul de $\overline{C_{n+1}}$: $t_0$
    \item Délai pour le calcul de $C_{n+1}$ (à partir de $\overline{C_{n+1}}$: $t_1$
    \item Délai pour le calcul de $\overline{S_n}$ (à partir de $\overline{C_{n+1}}$: $t_2$): $t_2$
    \item Délai pour le calcul de $S_n$ (à partur de $\overline{S_n}$): $t_3$
\end{itemize}

Pour la suite, on pose $C_{ox} = \SI{62.5}{aF}$

\subparagraph{Calcul de $t_0$}
Dans le pire cas, chaine de 3 NMOS de largeur 3.
Capacité du noeud $\overline{C_{n+1}}$:
\begin{itemize}
\item Sortie des NMOS: $3/2 C_{ox}$
\item PMOS (de precharge): $C_{ox}$
\item Entrée de l'inverseur: $9/2 C_{ox}$
\item Entrée du PMOS du circuit de la somme: $9 C_{ox}$
\end{itemize}

Capacité totale: $C_{out0} = 16 C_{ox}$.

On a donc
\[t_0 = (0.35\cdot 3^2 C_{ox} + 0.7 C_{out0})\SI{34}{k\ohm} = \SI{30.5}{ps}\]

\subparagraph{Calcul de $t_1$}
Capacité du noeud de sortie $C_{n+1}$: on considère que le carry-out est
connécté au carry-in d'un full adder.
\begin{itemize}
\item Capacité de sortie de l'inverseur $3C_{ox}$
\item NMOS: $9/2 C_{ox}$
\item PMOS: $(6 + 15/2) C_{ox}$
\end{itemize}

Capacité totale: $C_{out1} = 39/2 C_{ox}$.

On a donc
\[t_1 = 0.7 \cdot \SI{34}{k\ohm} C_{out1} = \SI{29.0}{ps}\]

\subparagraph{Calcul de $t_2$}
Dans le pire cas, chaine de 5 PMOS de largeur 5.
Capacité du noeud $\overline{S_n}$:
\begin{itemize}
\item NMOS: $C_{ox}/2$
\item PMOS ($\overline{C_{n+1}}$): $3C_{ox}$
\item PMOS ($C_n$): $5C_{ox}$
\item Entrée de l'inverseur: $9/2 C_{ox}$
\end{itemize}
Capacité totale: $C_{out2} = 13 C_{ox}$

On a donc
\[t_2 = (0.35\cdot 5^2\cdot 2 C_{ox} + 0.7 C_{out2})\SI{34}{k\ohm} = \SI{56.5}{ps}\]

\subparagraph{Calcul de $t_3$}
Sortie d'un inverseur:
\[t_3 = 0.7 \cdot 3 \cdot C_{ox} \SI{34}{k\ohm} = \SI{4.5}{ps}\]

\subparagraph{Délai total}
Au total, le délai pour un additionneur de 2 bits est
\[t_{FA2} = t_0 + t_1 + t_0 + t_2 + t_3 = \SI{151}{ps}\]

\paragraph{Fréquence maximale}
Comme seulement la moitié de le période d'horloge est disponible pour
effectuer les calculs (l'autre moitié est utilisée pour le \emph{prefetch}),
la période d'horloge minimale est \SI{302}{ps}. La fréquence maximale est
donc \SI{3.31}{GHz}.

\paragraph{Additionneur de 32 bits}
Le délai est
\[t_{F32} = 31 (t_0 + t_1) + t_0 + t_2 + t_3 = \SI{1.94}{ns},\]
donc la période d'horloge minimale est de \SI{3.87}{ns}, donc la
fréquence maximale est \SI{258}{MHz}.
\end{document}

