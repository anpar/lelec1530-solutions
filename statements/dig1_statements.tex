\subsection*{Exercice 1 : modèles digitaux}

\begin{figure}[!htbp]
   \centering
   \includegraphics[width=10cm]{figure/fig7-1.png}
   \caption{Exercice 1}
   \label{fig7-1}
\end{figure}

Calculez les délais à 50\% des transitions symbolisées sur la figure~\ref{fig7-1}
pour des transistors à canaux court en technologie 50nm avec \cox=62.5aF/$\mu m^2$,
$R_n$=34k$\Omega\mu$m et $R_p$=68k$\Omega\mu$m.
Les tailles des NMOS valent $W=10\mu$m et $L=1\mu$m tandis que celles des PMOS valent
$W=20\mu$m et $L=1\mu$m.

\subsection*{Exercice 2 : inverseur CMOS}

On considère une chaîne d'inverseurs CMOS identiques avec ses pistes d'interconnexion
en technologie 0.18$\mu$m, comme représenté à la figure~\ref{fig7-2}. Voici les paramètres
technologiques :

\begin{center}
$
	\begin{array}{l l l}
		L_{min} = 0.18 \mu m 	& V_{DD} = 1.8 V 		& V_{t0,n} = |V_{t0,p} | = 0.5 V \\
		\mu_n = 0.04 m^2/Vs 	& \mu_p = 0.02 m^2/Vs	& C_{ox} = 5 fF/\mu m^2 \\
		C_{rout} = 10fF			&						& \\
	\end{array}
$
\end{center}

\begin{figure}[!htbp]
   \centering
   \includegraphics[width=10cm]{figure/fig7-2.png}
   \caption{Exercice 1}
   \label{fig7-2}
\end{figure}

\emph{Analyse DC}
\begin{enumerate}
	\item Pour une transition infiniment lente à l'entrée de I1, déterminez le régime de
	fonctionnement de M1 et M2 ainsi que la tension \vout en fonction de \vin.
\end{enumerate}

\emph{Délai de propagation}
\begin{enumerate}
	\setcounter{enumi}{1}
	\item Comme estimation du délai de propagation ($t_p$) de I1, on vous demande de
	calculer le temps de descente de \vout à 50\% lors d'un flanc montant sur \vin
	($t_{PHL}$). Pour ce faire, remplacer M1 par une résistance équivalente $R_n$.
	%\item Comme estimation du délai de propagation ($t_p$) de I1, on vous demande de
	%calculer le temps de descente de \vout à 50\% lors d'un flanc montant sur \vin
	%($t_{PHL}$). Pour ce faire, remplacer M1 et M2 par des éléments plus simples
	%(résistance, sources) en fonction du régime de fonctionnement. On suppose que
	%le flanc sur \vin est idéal et on néglige les capacités parasites.
	%\item On souhaite simplifier le calcul du délai de propagation en remplaçant M1
	%par une résistance équivalente $R_{eq}$ qui donne la même valeur de délai que
	%celle trouvée précédemment. Pour ce faire, on vous demande de trouver un facteur
	%de fitting k pour exprimer $R_{eq}$ en fonction de la résistance en mode linéaire
	%:\\ $R_{eq}$ = k $R_{lin}$ .
\end{enumerate}

\emph{Scaling de la technologie CMOS}
\begin{enumerate}
	\setcounter{enumi}{2}
	\item Répeter le point précédent pour les paramètres ci-dessous.
	%\item On considère cette fois une technologie 65nm avec les paramètres ci-dessous.
	%Calculer le délai sur base de la formule obtenue au point c). Négliger les capacités parasites.

	\begin{center}
	$
		\begin{array}{l l l}
			L_{min} = 65 nm 		& V_{DD} = 1.2V 		&V_{t0,n} = |V_{t0,p}| = 0.4V \\
			\mu_n = 0.05 m^2/Vs 	& \mu_p= 0.025 m^2/Vs	& C_{ox} = 7 fF/\mu m^2 \\
			C_{rout} = 2fF			&						& \\
		\end{array}
	$
	\end{center}
\end{enumerate}

\subsection*{Exercice 3 : Logique de passage}
On considère les 3 implémentations d'un multiplexeur 2 vers 1 de la figure~\ref{fig7-3} :
transmission gate, pass gate et static CMOS gate. La fonction logique associée est :
Y = (Sel.A) + ($\overline{Sel}$.B) et la technologie considérée est une technologie
0.18 $\mu$m avec les paramètres suivants:

\begin{center}
$
	\begin{array}{l l l}
		L_{min} = 0.18 \mu m 	& V_{DD} = 1.8 V 		& V_{t0,n} = |V_{t0,p} | = 0.5 V \\
		\mu_n = 0.04 m^2/Vs 	& \mu_p = 0.02 m^2/Vs	& C_{ox} = 5 fF/\mu m^2 \\
		C_{rout} = 10fF			&						& \\
	\end{array}
$
\end{center}

\begin{figure}[!htbp]
   \centering
   \includegraphics[width=9cm]{figure/fig7-3-1.png}
	 \includegraphics[width=9cm]{figure/fig7-3-2.png}
	 \includegraphics[width=9cm]{figure/fig7-3-3.png}
   \caption{Exercice 1}
   \label{fig7-3}
\end{figure}

\begin{enumerate}
	\item Déterminer les tensions des niveaux logiques haut ($V_{OH}$) et bas ($V_{OL}$) de
	la sortie Y pour chacun de ces multiplexeurs.
	\item Calculez le temps de décharge à 50\% ($t_{PHL}$) de la capacité $C_{load}$ par
	l'inverseur de l'étage suivant, suite à un flanc montant à la sortie Y. Considérez une
	résistance équivalente pour les transistors passant du régime saturé au linéaire de
	$R_{eq}$ = 2.5 $R_{lin}$ et négliger toutes les capacités parasites.
\end{enumerate}